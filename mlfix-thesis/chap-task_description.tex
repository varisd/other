\chapter{Task definition}
\label{chap:task_descr}
%TODO: trosku si pohrat se section/subsection (co ma byt co)

In this chapter we will present results of the closer inspection
of the available training data and describe our process of developing
the MLFix post-editing component.

%issues:
% - what to classify
% - how to correct
% - what are the training instances
% - 

% TODO: opravdu uz jsme to popsali?
As we already described, we decided to approach the post-editing problem as a classification
task. Basically, the main idea is to identify words with incorrect surface form,
assign a new morphological categories to these words (e.g. via new POS tags, Interset
categories) and generate a new surface form. Naturally, this raises several issues
that have to be resolved, mainly:
\begin{itemize}
    \item What instances should we extract from our training data?
    \item Which of the extracted instances should we mark as incorrect, as opposed
        to instances that represent words with correct surface from?
    \item What features should we extract for each instance?
    \item How should we apply the trained classifier? Should we apply it
        on each word in a sentence or should we identify the incorrect instances
        first?
    \item What morphological categories should be predicted by our classifier?
\end{itemize}

Another important question is how should we measure the quality of our solution.
Obviously, we cannot base the quality of our system on the performance of the
classifier itself since even a really well performing classifier can have only a small
impact on the edited sentences (e.g. if we choose to predict morphological categories
that have very small impact on the final surface form). Still, the standard metrics
used for classifier evaluation, such as accuracy, precision, recall, can be helpful
during the hyperparameter tuning.

In the end, however, our main goal isn't producing a well performing classifier, but producing
more fluent, grammatically correct sentences.
Naturally, for this purpose, the human evaluation of the post-edited sentences is the best choice as far
as reliable judgement goes, but it is also very costly and we usually need a more efficient
method during system developement. Therefore we rely on the widely used BLUE score and
probably even more suitable translation error rate (TER) metric. Relying on these methods alone
has however few drawbacks which we will address in the later in this chapter.

\section{Task definition}

As we already mentioned, our goal is assigning new surface form to the incorrect instances
in the MT translated text via newly predicted POS tags. For morphologically rich languages,
this can be quite inefficient due to large sparse tagsets. Also, most of the time,
the incorrect surface form of the word is only a result an incorrect value in a small
subset of morphological characteristics of the word (e.g. wrong case, number, gender).
These reasons are why we decided to use the Intereset categories instead\footnote{It is
not necessary for Czech, since the Czech positional tagset allows us to modify only necessary
parts of the POS tag. However, other languages do not provide similar tagset, therefore
Interset might be more feasible representation.}.

Next, we decided how our post-editing system should work. We settled on three possible scenarios:
\begin{enumerate}
    \item Have one completely general classifier that is applied on each word in the sentence (some
        words might be ommited, by hand crafterd rule, e.g. ignore classes that don't flect).
    \item Identify the incorrect instances first by a separate classifier. Apply the second classifier
        on the marked instances and predict new morphological categories.
    \item Identify the incorrect instances same way as in the previous scenario however, this time also choose a classifier that
        should be applied in the second step. The choice can be simply deterministic (e.g. via general POS), or stochastic.
\end{enumerate}

The first scenario can be quite attractive since it is very easy to implement and requires only one model.
The problem is that requirements for such model might be way too big and be difficult to satisfy. Another
problem might be an unbalanced training dataset since most of the training examples will represent
\equo{morphologically correct} (as far as fluency of the translated text is concerned)
wordforms where predicting a new morphological categories isn't desired. Therefore
producing a well performing model can be real nightmare.

We have found the second scenario to much more plausible and it was our main focus during this research.
It requires simple binary classifier for the first step and one multiclass (or possibly multitask\footnote{We
define a multitask classifier as a classifier combining two classification tasks together (e.g. predicting new number and num case).
The tasks are separate, however the classifier takes the possible relation of the two tasks into account.}) classifier.
When training the binary classifier (which identifies the words that need to be corrected) we still have to face
the issue of an unbalanced training dataset, on the other hand, the classifier assigning new morphological categories
can be simply trained only on the incorrect istances.

The third scenario seemed reasonable since we usually want to modify different morphological categories for different POS.
However due to the results of the data analysis we didn't find it feasible to implement for time being.

\subsection{Oracle classifier}

Now that we have outlined the post-editing subtasks and assigned hypothetical classifiers 
to solve each one of them, there are following two issues left to resolve: choice of the
training instances and choice of the classifier targets. Since we only focus on correcting the morphological
errors generated by the MT system and we ignore the lexical errors completely, we have to be carreful
when picking the example instances for our classifier. For this matter, we found it very helpful to
use a \equo{fake} classifier for this task, that we simply call \pojem{Oracle}.

The basic idea behind the Oracle classifier is that it has access not only to the
source sentences and the MT output like our production classifier but also to the \equo{correct} answers
contained in the \pojem{reference translations}/\pojem{post-edited sentences}. The most important
task of the Oracle is to help us to observe whether the suggested definition of \pojem{training instances} and
moreover the definition \pojem{incorrect instances} (i.e. the instances, which require post-editing)
has a potential of improving the MT output (in case a perfect classifier is available to us).

We decided to extract \pojem{training instance} for each word in our data that met all of the following criteria:
\begin{itemize}
    \item the \pojem{MT-node lemma} IS\_EQUAL to the \pojem{REF-node lemma},
    \item the \pojem{MT-parent\_node} IS\_DEFINED AND\newline{}
        the \pojem{MT-parent\_node} IS\_NOT\_ROOT,
    \item the \pojem{REF-parent\_node} IS\_DEFINED AND\newline{}
        the \pojem{REF-parent\_node} IS\_NOT\_ROOT,
    \item the \pojem{SRC-node} IS\_DEFINED,
    \item the \pojem{SRC-parent\_node} IS\_DEFINED AND\newline{}
        the \pojem{SRC-parent\_node} IS\_NOT\_ROOT,
\end{itemize}
We settled for these criteria for the following reasons: we want to extract only instances that have true predictions
available (presence of the aligned REF-node), we want to ignore \equo{incorrect} instances related
to lexical errors (equality of the SRC-node and REF-node lemmas) and finally, we want to make sure that
enough relevant tree-based context information will be extracted. We haven't modified the definition
of the training instance much during the research, but we were experimenting a with the different definitions
of the \pojem{incorrect instance} (i.e. instance, that has incorrect surface form).

We have tried three different definitions of \pojem{incorrect instance}, with following set of conditions:
\begin{enumerate}
    \item the \pojem{MT-node form} IS\_NOT\_EQUAL to the \pojem{REF-node form},
    \item the \pojem{MT-node form} IS\_NOT\_EQUAL to the \pojem{REF-node form} AND the \pojem{MT-parent form} IS\_EQUAL to the \pojem{REF-node form},
    \item the \pojem{MT-node form} IS\_NOT\_EQUAL to the \pojem{REF-node form} AND (the \pojem{MT-parent form} IS\_EQUAL to the \pojem{REF-node form}
        OR the \pojem{MT-parent form} is marked as INCORRECT)
\end{enumerate}
All definitions work in the context of the \pojem{training instance}. In any case, if the conditions
are not met, the training instance is marked as correct and should be left unchanged by the classifier.
These definitions cover both instances that should be changed (\pojem{correct} vs. \pojem{incorrect} instance)
and how the incorrect instances should be modified (values from the REF-node). The surface forms
generated with the use of these training instances should be indentical to those of the reference.
This fact has one major drawback when defining the incorrect instances:
the quality of the Oracle classifier (and therefore the \equo{best}
possible result) cannot be reliably measured by the automatic n-gram based metrics, because the post-edited
sentences will always have same or better score than the MT output. Therefore manual evaluation is required
to some extent.

The first method marked about one-tenth of the training instances as incorrect. When used in combination
with the Oracle classifier the post-edited sentences have shown moderate improvement in the automatic metrics.
However, during the observation of the sentences we noticed many modifications such as~\Eref{ex-oracle-noref}.
Clearly, even though the wordform of \samp{místo} has been changed to match the reference,
the governing verb \samp{mít} requires its subordinate to be in the dative case, therefore the change
has actually introduced a new grammatical error into the sentence and thus worsened
the fluency of the translated sentence. Additionally, by changing the case of the
case of the word \samp{místo} the already correct agreement with the subordinate adjectives
\samp{poslední} and \samp{volná} was also broken making the result even worse. On the other hand,
the wordform in the reference sentence is correct because the governing word of \samp{míst}
is a noun (\samp{pár}) instead of a verb resulting in a genitive case of the word.

\begin{myexample}
    \small
    \catcode`\-=12
    \begin{center}
    \begin{tabular}{|l|p{0.75\textwidth}|}
    \hline
    \textbf{Source:}  &  \textbf{We have the last few vacancies for New Year's Eve and Christmas.}  \\
    \hline
    SMT output:  &  \textit{Máme} \textit{poslední} \textit{volná} \textbf{místa} na Silvestra a Vánoce.  \\
    \hline
    Gloss:  &  We \textit{have} the $\mathit{last_{dative}}$ $\mathit{few_{dative}}$ $\mathbf{vacancies_{dative}}$ for New Year's Eve and Christmas.  \\
    \hline
    Oracle output:  &  \textit{Máme} \textit{poslední} \textit{volná} \textbf{míst} na Silvestra a Vánoce.  \\
    \hline
    Gloss:  &  We \textit{have} the $\mathit{last_{dative}}$ $\mathit{few_{dative}}$ $\mathbf{vacancies_{genitive}}$ for New Year's Eve and Christmas.  \\
    \hline
    Reference:  &  Na Silvestra i na Vánoce \textit{máme} \textit{posledních} \textit{pár} \textit{míst}.  \\
    \hline
    \end{tabular}
    \label{ex-oracle-noref}
    \end{center}
\end{myexample}

The previous example has shown us that when identifying incorrect instances some additional information
about the surrounding members of the sentence is required.
We have seen that by slightly altering governing nodes, e.g.
by only choosing a different lexical translation of the source node or by choosing a completely different expression,
the equality between the surface form of the MT word and its reference counterpart cannot
be enforced without harming the quality of the MT sentence.
Therefore, we tried introducing additional constraint to correctly identify candidates for post-editing.
To mark and instance as incorrect, not only the ref-node form has to be different of its aligned
ref-node form. Note, that we considered checking only for the parent node lemmas at first, but this constraint
was too soft and was not able eliminate some of the previous errors.
The main motivation behind this constraint is to identify at least some agreement and possibly valency errors
without too much of a language-specific insight. We assume that if the surface form of the mt-node and ref-node
differ while their parent node is identical, there is a high chance that the mt-node's surface form is
incorrect and the reference node has the right correction\footnote{This, of course, depend on the quality of the dependency
tree produced by the parser. However, since we use different parsing methods for mt-side (projection of the src-tree) and
the reference (dependency parser) we expect that the number of false positives in our training data will be low.}.


This additional constraint helped us to remove large number of false positives and produced training examples
such as~\Eref{ex-oracle-parentref}. In this example we can see, that the noun \samp{život} was correctly changed
to \samp{života} because of the valency frame of the verb \samp{sdílet}. However, we can see, that this constraint
might be too strict because the adjective \samp{akademického} was left unmodified even though by changing its
governing nound \samp{život}, the agreement that was present in the MT output should be preserved after the
post-editing. This was not done because the surface forms of the governing nodes in the MT sentence and the reference sentence
(\samp{život} vs. \samp{života}) did not match. We observed this kind of false negative quite often which
led us to introducing one additional constraint.


%TODO: parsing reference

\begin{myexample}
    \small
    \catcode`\-=12
    \begin{center}
    \begin{tabular}{|l|p{0.75\textwidth}|}
    \hline
    \textbf{Source:}  &  \textbf{…where he acknowledged the "wonderful people" he shared his academic life with.}  \\
    \hline
    SMT output:  &  …kde potvrdil, že je "skvělí lidé" \textit{sdílel} jeho akademického \textbf{života}.  \\
    \hline
    Gloss:  &  …where he acknowledged, that is "wonderful people" he \textit{shared} his academic $\mathbf{life_{genitive}}$.  \\
    \hline
    Oracle output:  &  …kde potvrdil, že je "skvělí lidé" \textit{sdílel} jeho akademického \textbf{život}.  \\
    \hline
    Gloss:  &  …where he acknowledged, that is "wonderful people" he \textit{shared} his academic $\mathbf{life_{dative}}$.  \\
    \hline
    Reference:  &  …do poděkování "skvělým lidem", s nimiž \textit{sdílel} akademický \textit{život}.  \\
    \hline
    \end{tabular}
    \label{ex-oracle-parentref}
    \end{center}
\end{myexample}

To soften the previous constraint in favor of not breaking the agreement of the subordinate nodes we decided
that if the previous constraint is not satisfied (possibly due to a difference between the mt-node surface form
and the ref-node surface form), the node in question should still be marked as incorrect if and only if
the governing node is marked as incorrect and the node's surface form is different form that of the reference
node. This is basically similar to post-editing the MT output dependency tree recursively from its
root to its nodes. Applying this new constraint we were able to produce training examples similar to %TODO example
%TODO okec examplu

It should be noted, that these constraints do not detect all possible morphological errors. For example,
because only node's parent is checked we effectively omit subject-verb agreement errors. This can be possibly
fixed in the future by additional constraints however, in the scope of this thesis, our main objective was
to find and evaluate a method for indentifying and correcting morphological errors that is as general as possible.
%TODO manual eval

It is also worth mentioning that even though the final constraint produced quite reasonable training examples
with little to no false positives, %TODO is it true?
there were still instances such as~\Eref{ex-oracle-ambig}, where it is difficult to decide whether the post-editing
helped to improve or harmed the fluency of the MT output.
In this example, we can see that the adjective \samp{osvětleného} was changed to \samp{osvětleném} because the
governing node \samp{baru}, being a part of the apposition with the \samp{Julep Room}, should
be in the preposition-noun agreement with the preposition \samp{v}. However due to mistranslation of the
named entity (\samp{Julep Room}) and wrong indication of the apposition relationship (missing comma),
the reader might consider the correction made by the Oracle less fluent or natural. On the other hand,
the meaning of the original MT output is quite different from the meaning in the source sentence.

As we have seen, the task of extracting meaningful and correctly anotated training instances for our system
 gets increasingly difficult with growing difference between the MT output and the reference sentences.
If provided output only from a poor MT systems, the task might be almost impossible. For this reason,
we tried another, little limited, method for creating and extracting the training instances.

\begin{myexample}
    \small
    \catcode`\-=12
    \begin{center}
    \begin{tabular}{|l|p{0.75\textwidth}|}
    \hline
    \textbf{Source:}  &  \textbf{…entertainment coordinator at The Julep Room, a dimly lit bar near Gautier, said…}  \\
    \hline
    SMT output:  &  …programový koordinátor \textit{v} Julep \textit{místnosti} \textbf{osvětleného} \textit{baru} u Gautiera, prý…  \\
    \hline
    Gloss:  &  …entertainment coordinator at The Julep \textit{room} a $\mathbf{lit_{geneitive}}$ $\mathit{bar_{genitive}}$ near Gautier, said…  \\
    \hline
    Oracle output:  &  …programový koordinátor \textit{v} Julep místnosti \textbf{osvětleném} \textit{baru} u Gautiera, prý…  \\
    \hline
    Gloss:  &  …entertainment coordinator $\mathit{at_{locative}}$ The Julep room a $\mathbf{lit_{locative}}$ $\mathit{bar_{locative}}$ near Gautier, said…  \\
    \hline
    Reference:  &  …koordinátor zábavy \textit{v} \textit{The Julep Room}, spoře \textit{osvětleném} \textit{baru} poblíž Gautier, řekl…  \\
    \hline
    \end{tabular}
    \label{ex-oracle-noref}
    \end{center}
\end{myexample}
\todo{ukazat i rozdilne dep. struktury u ex. 4.3?}

\section{Depfix reference}

The constraint based method achieves best results when combined with data, where reference sentences
(or in this case rather human post-edited sentences) are as close to the MT output as possible.
For this reason, we decide to examine training data, where the reference sentences are \equo{syntetic},
they are neither produced by a human translator or created by a human post-editing process.
The best tool for this task is quite naturally Depfix.

Main focus of the Depfix post-editing tool is correcting morphological errors (aside from some more
frequent lexical errors, e.g. missng relfexive verbs and fake named entities) which usually 
results in post-edited sentences that are almost similar to the MT output. Aside from that,
the morphological changes (even though they are a result of an applied set of rules) made
by Depfix are quite identical to the changes we are trying to teach our statistical component.

As a result, we created a new datasets by applying Depfix on the available bilingual data and
using Depfix output in place of reference sentences. We then extracted the training instances
similar way as with the regular data. We have also run our Oracle classifier on the data with
the resulting sentences copying the Depfix output most of the time.

This method seems quite viable for the task of identifying incorrect instances and learning the
right correction method. It should be also aplicable on various MT systems because, as Rosa\todo{ref?}
previously reported, Depfix was able to improve the quality of various systems to some degree.
The downside is, that this method is limited only to the Czech language since the Depfix was
created with the aim to post-edit English-Czech machine translation. If we were able to
reuse the models trained on the English-Czech data for another language pairs, this method
would become quite valuable. 

\section{Feature extraction}

%TODO: tabulky
% # of extracted instances (parents eq, !parents eq...)
% automatic statistic of extracted/filtered features (variance filter)
% popsat ruzne mnoziny featur + velikosti (extracted filtered)
% add aggreement features


When we were designing the initial set of features we chose the following
strategy: extract as many distinct features based on the available node information
and later reduce the features by one or several methods (manual, or stochastic)
for features selection.

Our feature set has a hierarchical structure. For each training instance we extract
information about the node, its parent, the aligned source node and aligned source node's
parent. For training purposes, we also extract information from the aligned reference
node. Note, that we ignore information about the parent of the reference node, we
use these information only for distinguishing the incorrect instances from the correct ones
according to the previously described aproach.

From each of these \samp{main} nodes inside the instance, we extract information
specific to them and to their close neighbourhood (e.g. their parent, grandparent,
preceding child, following child, preceding sibling, following sibling). We
also extract information about the number of preceding and following children,
direction of the edge coming from the node's parent and finally Interset representation
of the POS tag. As a default, if a value is not defined we use an empty string instead.
We are allowed to do that because we treat every feature as a set of discrete values.

We had also tried extracting the lemmas however, it resulted in a
large growth of an already quite big feature space, so we dropped the idea.
In the future, it might be interesting to use at least limited amount of lemmas in our
feature set (most frequent lemmas) and observe how they can affect the performance of the classifiers.

This way, we extracted around 1500 initial features. Of course, this was general feature
set that needed to be further processed within every classification task. Also, the method
by which we chose the initial features resulted in many of the features having zero
variance. Those features we removed.

% TODO: statistiky

