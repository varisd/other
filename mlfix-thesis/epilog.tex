\chapter{Conclusion}
%\chapter*{Conclusion}
%\addcontentsline{toc}{chapter}{Conclusion}

In this thesis, we have present MLFix, an automatic post-editing tool focusing on statistical
post-editing of the incorrect morphology in the machine translation output. The system was
developed as a successor of a rule-based Depfix system, with aim to generalize some of its
rules to a stochastic model which can be applied across languages.

During the developement, we had to find a compromise between the level of language independence
and overall usefulness fo the system. In the end, we have chosen a unique approach to the problem of correcting
the morphology by solving a two-step classification task: error detection and morphology prediction.
We have faced a problem of automatic identification of correct/incorrect training instances
which we have solved with a fairly effective heuristic. Still, further refinement of the
training data extraction method is desired in the future.

Out of the two classification tasks, the morphology prediction proved to be much easier.
The resulting models, while being very good at predicting simple categories (e.g. morphological case)
manifeste much lower individual performance with increasing task complexity. However when
combined together, they performed quite well.
Also, since we have used only a really small datasets for model training, we think that these models
can be improved in the future by increasing the training data or adding additional features.
This claim is supported by a model trained on a much larger Autodesk dataset, which achieved really good
in-domain performance even when it was trained to classify multitask problem (prediction of case-gender-number)
Furthermore, we think that
these models have potential use even in different fields of application, e.g. as a part of automatic
correction suggestion in a human post-editing framework.

The task detecting targets for our morphology prediction tool, became main a hurdle
during the developement of MLFix. Aside from the correct identification of the training instances
in our data, there was also an issue with highly unbalanced training set which we partially resolved
by upsampling the minority class and filtering out instances from the \equo{correct} sentences.
Even though the resulting models performance seemed unsatisfactory at first, they performed resonably
well during final evaluation as far as precision of the resulting system was concerned. The weaker
side was fairly low overall impact on the MT output. As far as future improvement goes the results
achieved during model training on the large Autodesk dataset suggest that the performance can still be
improved simply by icreasing the amount of our training data.

As we are mentioning using larger training datasets in the future, in the scope of this thesis we have
focused mainly on investing human-postedited data that are, at the moment, available in much smaller
volumes than data containing reference translation. However, we have tried training few models on smaller
datasets with reference sentences instead of human-posteditations. The resulting models still performed
fairly well if the reference sentences were reasonably similar to the MT output. Lastly, we have
also examined data created by replacing the human-postedits with Depfix output resulting in reliable
source of training data. On the other hand these data tend to be much more sparse (as a result of Depfix
impact on the MT output) and the method is currently restricted to English-Czech language pair only.
However, if we can achieve adapting Czech models for other languages in the future, this method might
become viable.

During the final evaluation the system performed well when measured with BLEU scoring metric.
These results were confirmed to some extent by a manual evaluation, however, we are still not
completely confident in the results and suggest evaluating MLFix on much larger scale in the future.
Surprisingly, MLFix was able to surpass Depfix when it was applied to the output of NMT system.
This result caught our attention and will be investigated closely in the near future. If confirmed
the application to the increasingly popular aproach to the machine translation might become valuable.

We have also evaluated performance of a modification aimed at correction German SMT output. We were
satisfied with the results during model developement, achieving results similar to Czech pipeline,
which confirms that the classification tasks themselves (as they were defined in the thesis) are
not language dependent. No special attention to manually modifying the feature set or choosing
different approach was required.

The results of the final evaluation were very poor. Aside from the morphology module applied separately,
German MLFix always worsened the MT output as the automatic metric have shown. However, we pointed out
several indications that this might be caused by a poor performance of the German inflection module
we use to generate new wordforms. We still have to investigate the matter further to confirm this hypothesis.
However, if confirmed, replacing (or improving) the module in the future might be the first and fastest
way to improvement.
Another goal is to investigate the German MT erros more thoroughly and adapt the German pipeline to the findings.

Even though we mainly focused on morphology correction in this thesis, MLFix can be further improved
in the future by introducing other statistical modules addressing addition MT errors. As an example
we mention a possible word reordering model because there were instances where words with new surface
forms predicted by MLFix still needed rearangement to fully utilize the modification. Due to the modularity
of Treex framework, introducing new improvements to MLFix is easy.
