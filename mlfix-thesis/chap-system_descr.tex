\chapter{System description}
\label{chap:system_descr}

In this chapter we are going to describe the main components
of the MLFix system and take a closer look at the suggested processing pipeline.

The main idea of the MLFix system is to analyze the provided input data providing
a set of input features 

\section{Processing pipeline}

The MLFix is, similarly to its predecessor, almost entirely implemented in the
Treex\cite{Popel:2010:TMN:1884371.1884406}\footurl{http://ufal.mff.cuni.cz/treex}
framework (formerly knownas TectoMT).
The framework was originally created as English-Czech hybrid translation system, combining
rule-based modules with the statistical models and using deep semantic language representation
for the sentence translation. However, due to its modularity, it is
now used for various tasks of natural language processing (NLP) across different
languages. The framework was built to support the methodology of the theory of Functional Generative Description\cite{Sgall1967}
and was adapted to support sentence representation in the Prague Dependency Treebank\cite{pdt20:2006}.
Mainly it supports the representation of the sentence on the different layers of abstaction defined in FGD: morphological layer,
analytical layer and a tectogrammatical layer.\footnote{Usually referred to as m-layer, a-layer and t-layer where prefixes m-, a- and t- are also used to refer to the objects on the corresponding layer of abstraction.}
Because the tools currently available for tectogrammatical layer analysis are available for
a limited set of languages (mainly Czech and English, the developement of others is in progress),
we decided to use only the a-layer (surface syntax) for data representation.

\subsection{M-Layer analysis}
\todo{a-layer analysis + okec, ze m-layer je v treexu sloucena s a-layer}
The MLFix analysis pipeline is derived from an existing Depfix pipeline
with a several modifications to make it easier to apply to different
target languages. The input data (source sentence + MT output aligned on the sentence level,
or additionaly reference sentences for extracting the training data)
are first read in parallel and stored into the Treex internal representation.
Both source side and MT side are tokenized by a rule-based tokenizer, each token is then
represented by a separate m-node.

Next, the lemmatization and part-of-speech (POS) tagging is performed on each m-node.
For both English and Czech, we use MorphoDiTa\cite{strakova14:morphodita}\footurl{http://ufal.mff.cuni.cz/morphodita}
tool for morphological analysis and tagging. MorphoDiTa is also used for Czech lemmatization.
For English lemmatization, we use a rule-based block implemented in Treex.
It is important to provide a POS tagger for the target language that supports
relatively fine-grained morphological tags because our goal is to correct the morphological
errors mainly through these tags.
It was reported by Rosa\fixme{znovu citovat diplomku?} that tagger produces significantly more
errors in morphological analysis of the Czech SMT outputs. In Depfix, this is covered
by a rule-based block that identifies these errors and changes the incorrect POS tag without
changing the surface form. We decided to omit this block and leave the issue to our statistical
component.



Last step we do in the scope of the m-layer is transformation of the POS tags into
a more general representation. Optionally we can apply a named-entity recognition
tool if one is available but it isn't mandatory. For English, we use the Stanford
Named Entity Recognizer (NER)\cite{Finkel:2005:INI:1219840.1219885}\footurl{http://nlp.stanford.edu/software/CRF-NER.shtml},
for Czech, we use a simple rule-based NER. 

\subsection{Interset}

Since there are usually different tagsets used across the individual languages,
often engineered for purposes of that specific language and with no standardized
tag representation,
we would be forced to modify our existing pipeline to some extent everytime
a new language would be introduced.
Therefore we decided to use Interset\cite{biblio:ZeReusableTagset2008}\footurl{https://ufal.mff.cuni.cz/interset}, an interlingua-based
representation of POS tags from various tagsets. To be able to use this
representation to represent tags from a given tagset, the decoding/encoding
module is required. However, lately the support for various tagsets spanning
through different languages started growing mainly due to the Universal Dependencies\footurl{universaldependencies.org} project.

After the transformation, in the following steps of the analysis, the choice
of a specific tagset becomes transparent because our blocks only
have to deal with one well-defined set of features.

\subsection{Word alignment}

For the next step, we create a word-level alignment between each sentence pair
using GIZA++\cite{och:ney:2000}. We make one-to-one word alignment where it is possible
through the use of the intersection symmetrization. This step helps us later with feature extraction
and with further processing of the target sentence.

In the process of training data extraction, we also create simple alignment between
the MT sentence and the reference sentence exploiting the forms, lemmas and tags
of the m-nodes. The alignment between the source sentence and MT ouptut is then also
projected to the reference sentence.

\subsection{A-Layer analysis}

After the m-layer analysis and word alignment, we perform a dependency parsing.
For English, we use Maximum spanning tree (MST) parser\cite{mcdonald:pereira:ribarov:hajic:2005}\footurl{http://sourceforge.net/projects/mstparser}
implemented in the Treex framework. For the SMT output, even though there might
be an existing dependency parser available for the target language, it is usually
trained on the data that do not contain errors. Therefore, it has usually significantly
lower performance when applied on the SMT output. Still research on Depfix have shown
that the knowledge of the dependency structure of the SMT output can provide additional
valuable information for identifying the grammatical errors\footnote{Actually, in the case of Depfix
, the information about the dependency structure is crucial for most of the fixing components
because e.g. the parent-child relationship is examined almost everytime.}, thus improving
the post-editor's performance.

For the time being we decided to build the dependency structures of the SMT output simply
by projecting the dependency structure of the source sentence onto the target side
using the word alignment we extracted in the preceeding step. The resulting structure
will most likely contain errors, but should be at least consistent thoughout our data.
To compensate the lower accuracy of the extracted dependency structure we perform
a dependency parsing of the reference sentences during training, if a proper parser
is available. For this purposes, we use the MST Parser for the Czech language as well.
We expect that the combination of dependency parser for the reference sentences and
a right choice of constraints applied when extracting training instances we should
avoid polluting our training data with instances containing misleading context information
due to an incorrect structure of the SMT dependency tree.

For the Czech SMT output, an implementation of the MST parser adapted for the SMT output
is already available\cite{biblio:RoDuUsingParallel2012}, however, so far we haven't done any experiments
to determine whether the dependency structures provided by the adapted parser (which
will most likely be more precise) influence the final performance of our system.

%\subsection{Training pipeline}

%We also do preprocessing of data we use for training our statistical component.
%The pipeline is similar to the processing pipeline with addition of processing
%the reference sentences. The reference sentences are processed in a same way
%as the MT output, however they are not analyzed on the a-layer, only a simple
%lemma based word alignment with the MT output is made.

\section{Statistical component}

After the data preprocessing, we extract all available features and
apply a trained statistical model. The features are extracted separetely for each node and passed
to the model. The model needs to accomplish these two goals:
\begin{enumerate}
    \item identify the candidates with incorrect morphology,
    \item fix the incorrect morphological features.
\end{enumerate}
Of course, these two steps can be split between multiple separate components (and models),
we describe the post-editing process in more detail later in chapter~\ref{chap:tuning}.

We decided to use Scikit-Learn\cite{scikit-learn}\footurl{biblio:RoDuUsingParallel2012} toolkit to train and execute our models since
it has easy-to-use and quite uniform interface, which allows us to try out different
machine learning ML methods simply by switching the model class. In Treex, we use a simple
wrapper to load and execute the trained Scikit-Learn model. If a support for a different ML implementation
is needed (e.g. VowpalWabbit\footurl{https://github.com/JohnLangford/vowpal_wabbit/wiki}\footnote{At the time of writing this thesis, there is already
a limited support for VowpaWabbit.}), this wrapper can be easily modified to suit the requirements.

\section{Wordform generation}

After we have finished identifying the morphologically incorrect words and assigning
them a new morphological categories we need to generate a new surface form reflecting
the changes we have made.
This can be done either by rule-based component or with
help of another statistical model. 

For Czech, we have used a morphological generator build upon the morphology of Hajič\cite{HajicHAB2004}
that has already been implemented within the Treex framework. For other languages,
when there was not another option available,
we have used Flect\cite{DBLP:conf/acl/DusekJ13}\footurl{https://ufal.mff.cuni.cz/flect}.
It is a language independent morphological generation tool also using the Scikit-Learn
models. The tools learns morphological inflection patters form an annotated corpora and
is should able to even inflect previously unseen words using lemma suffixes as features
and predicting the difference between the lemma and the form.

\section{Language independence}

From the previous description, we can notice that MLFix still depends on several
language specific tools. It is quite dependent on the capability of the source
language analysis (in our case English), because we do not only need a POS tagger but also a dependency
parser. We however assume that the source sentences are usually grammatical correct
so it is much easier to provide the required tools than their modified versions
targeted at the MT output.

We also require a specific decoder of the source and target POS tags into the Interset
structure but Interset already covers large variety of the most widespread tagsets available
and its support is still growing.

Finaly, aside from a language specific post-editing models (which we do not expect
to be reusable across different languages) we require a module that regenerates
the correct word form (either from the \samp{lemma+tag} or \samp{lemma+Interset features combination}).
A state-of-the-art tool might not be explicitly available for every language but if no
other option is provided we can use a statistical form generator, in our case Flect.
