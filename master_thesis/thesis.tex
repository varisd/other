%% The main file. It contains definitions of basic parameters and includes all other parts.

%% Settings for single-side (simplex) printing
% Margins: left 40mm, right 25mm, top and bottom 25mm
% (but beware, LaTeX adds 1in implicitly)
\documentclass[12pt,a4paper]{report}
\setlength\textwidth{145mm}
\setlength\textheight{247mm}
\setlength\oddsidemargin{15mm}
\setlength\evensidemargin{15mm}
\setlength\topmargin{0mm}
\setlength\headsep{0mm}
\setlength\headheight{0mm}
% \openright makes the following text appear on a right-hand page
\let\openright=\clearpage

%\renewcommand{\baselinestretch}{1.5} 

%% Settings for two-sided (duplex) printing
% \documentclass[12pt,a4paper,twoside,openright]{report}
% \setlength\textwidth{145mm}
% \setlength\textheight{247mm}
% \setlength\oddsidemargin{14.2mm}
% \setlength\evensidemargin{0mm}
% \setlength\topmargin{0mm}
% \setlength\headsep{0mm}
% \setlength\headheight{0mm}
% \let\openright=\cleardoublepage

%% Character encoding: usually latin2, cp1250 or utf8:
\usepackage[utf8]{inputenc}

%% Further useful packages (included in most LaTeX distributions)
\usepackage{amsmath}        % extensions for typesetting of math
\usepackage{amsfonts}       % math fonts
\usepackage{amsthm}         % theorems, definitions, etc.
\usepackage{bbding}         % various symbols (squares, asterisks, scissors, ...)
\usepackage{bm}             % boldface symbols (\bm)
\usepackage{graphicx}       % embedding of pictures
\usepackage{fancyvrb}       % improved verbatim environment
%\usepackage{natbib}         % citation style AUTHOR (YEAR), or AUTHOR [NUMBER]
\usepackage[round]{natbib}
\usepackage[nottoc]{tocbibind} % makes sure that bibliography and the lists
			    % of figures/tables are included in the table
			    % of contents
\usepackage{dcolumn}        % improved alignment of table columns
\usepackage{booktabs}       % improved horizontal lines in tables
\usepackage{paralist}       % improved enumerate and itemize
\usepackage[usenames]{xcolor}  % typesetting in color

\usepackage{multirow}
%\usepackage{obo-cite}

\usepackage{listings}
%\usepackage{breakcites}

\usepackage{threeparttable}

%%% Basic information on the thesis

% Thesis title in English (exactly as in the formal assignment)
\def\ThesisTitle{Automatic Error Correction of Machine Translation Output}

% Author of the thesis
\def\ThesisAuthor{Dušan Variš}

% Year when the thesis is submitted
\def\YearSubmitted{2016}

% Name of the department or institute, where the work was officially assigned
% (according to the Organizational Structure of MFF UK in English,
% or a full name of a department outside MFF)
\def\Department{Institute of Formal and Applied Linguistics}

% Is it a department (katedra), or an institute (ústav)?
\def\DeptType{Institute}

% Thesis supervisor: name, surname and titles
\def\Supervisor{RNDr. Ondřej Bojar, Ph.D.}

% Supervisor's department (again according to Organizational structure of MFF)
\def\SupervisorsDepartment{Institute of Formal and Applied Linguistics}

% Study programme and specialization
\def\StudyProgramme{Master of Computer Science}
\def\StudyBranch{Mathematical Linguistics}

% An optional dedication: you can thank whomever you wish (your supervisor,
% consultant, a person who lent the software, etc.)
\def\Dedication{%
I would like to thank my supervisor, RNDr. Ond\v{r}ej Bojar, Ph.D., for valuable
advice and support during writing this thesis. I would also like to thank
Mat\v{e}j Trojan for helping me with the evaluation of Czech MLFix output. Finally,
I would like to thank Ladislav Valkovi\v{c}, Ph.D., Mgr. Ond\v{r}ej Du\v{s}ek and Radek S\'{i}le\v{s} for
helping me with evaluating German MLFix output and providing me information about the language.

This thesis is dedicated to them.
}

% Abstract (recommended length around 80-200 words; this is not a copy of your thesis assignment!)
\def\Abstract{%
We present MLFix, an automatic statistical post-editing system, which is a spiritual successor of the rule-based
system, Depfix. The aim of this thesis was to investigate the possible approaches to automatic identification
of the most common morphological errors produced by the state-of-the-art machine translation systems and
to train sufficient statistical models built on the acquired knowledge.
We performed both automatic and manual evaluation of the system and compared the results with Depfix. 
The system was mainly developed on the English-to-Czech
machine translation output, however, the aim was to generalize the post-editing process so it can be
applied to other language pairs. We modified the original pipeline to post-edit English-German machine
translation output and performed additional evaluation of this modification.
}

% 3 to 5 keywords (recommended), each enclosed in curly braces
\def\Keywords{%
{automatic post-editing,} {machine translation,} {supervised machine\newline
learning,} {natural language processing,} {Treex}
}

%% The hyperref package for clickable links in PDF and also for storing
%% metadata to PDF (including the table of contents).
\usepackage[pdftex,unicode]{hyperref}   % Must follow all other packages
\hypersetup{breaklinks=true}
\hypersetup{pdftitle={\ThesisTitle}}
\hypersetup{pdfauthor={\ThesisAuthor}}
\hypersetup{pdfkeywords=\Keywords}
\hypersetup{urlcolor=blue}


%%%% Custom Definitions %%%

\newcommand{\fixme}[1]{\textcolor{red}{FIXME: (#1)}} % macro for fixme entries
\newcommand{\todo}[1]{\textcolor{blue}{TODO: (#1)}} % macro for todo entries

\def\samp#1{``\textit{#1}''}
\def\pojem#1{\textit{#1}}
\def\code#1{\texttt{#1}}


\def\Tref#1{Table~\ref{#1}}
\def\Fref#1{Figure~\ref{#1}}
\def\Eref#1{Example~\ref{#1}}
\def\Sref#1{Section~\ref{#1}}
\def\Cref#1{Chapter~\ref{#1}}
\def\Mref#1{Formula~\ref{#1}}
\def\equo#1{``#1''}
\def\notion#1{{\emph{#1}}}
\def\perscite#1{\newcite{#1}}
\def\parcite#1{\cite{#1}}
\def\footurl#1{\footnote{\url{#1}}}
\def\hash{\#}
\def\tilda{\~{}}


%%%%%%%%%%%%%%%%%%%%%%%%%%%

% Definitions of macros (see description inside)
\include{macros}

% Title page and various mandatory informational pages
\begin{document}
\include{title}

%%% A page with automatically generated table of contents of the master thesis

\setcounter{page}{1}
\tableofcontents

%%% Each chapter is kept in a separate file
\chapter{Introduction}
%\chapter*{Introduction}
%\addcontentsline{toc}{chapter}{Introduction}

In this thesis we are presenting a statistical post-editing tool MLFix
based on its rule-based predecessor Depfix\cite{depfix:2014}
. We aim to use statistical machine learning methods to generalize the subset of Depfix rules
with focus on creating fairly language-independent automatic post-editing (APE) tool.
Our main goal is to find a reasonable tradeoff between the amount
of linguistic knowledge gathered from the input data and the dependency
on the language-specific third-party analysis tools, to specify the
machine learning (ML) task which our APE component should accomplish and to train
a sufficient ML model for correcting the machine translation (MT) output.
The system was developed using English-Czech language pair, however, we also present
a preliminary results we gathered from the English-German language pair experiments.

\section{Task Motivation}

Even though the current state-of-the-art MT systems have been gradually improving in the
past few decades, they are still not perfect. Currently the most popular statistical machine
translation (SMT) systems can be fairly effective even without any or a little initial
linguistic knowledge about the concerned language pair given they have access to sufficient
amount of parallel data. However, when translating into the morphologically rich languages,
the data sparsity increases rapidly and such systems quickly begin to introduce grammatical
errors into the translated sentences and worsen the overall fluency of the translation.

Still, these systems can be of help to the human translators since it is usually less costly
to have even partially incorrect translations provided by the SMT system and have
human translator correct them then to translate the source text manually.
The human post-editors can however still be a quite costly so naturally there have been
attemts to automate this process with automatic post-editing tools.

\section{Related Work}

Up to this date there have been various approaches to the task of automatic
error correction of the machine translation output, each with a different level of success.

The very first attempts in the field of automatic post-editing\cite{simard2007rule}
focused on applying phrase-based statistical machine translation (PB-SMT)
system on the output of the rule-base machine translation. The system was trained on a monolingual
bitextual data containing the MT output as a source sentence and a reference translation as
the target sentence. This phrase-based automatic post-editor (PB-APE) helped to significantly improve
the performance of the rule-based MT system in question.
There were also attempts to apply the post-editing component
on a phrase-based translation system but the combined system performed slightly worse in comparison
with the standalone SMT system.

Further experiments with PB-SMT $+$ PB-APE were done by B\'{e}chara et al.\cite{bechara:2011}
on English-French and French-English translation, where significant improvements were reported
for the latter translation direction. They improved the design suggested by Simard et al. by
creating the purely statistical pipeline. The PB-APE was then expanded by adding additional
context information about the source sentence to SMT-generated output (used as an input for
the PB-APE) reporting further improvement in performance. However, the presented system failed to
improve the MT output in the opposite translation direction (English-French).

These previous attempts can be considered purely statistical since only a little ammount of linguistic
knowledge about the concerned languages was used during developement. To gain further insight into
the task of automatic post-editing, B\'{e}chara\cite{bechara:master} and later Rosa\cite{biblio:RoAutomaticpostediting2013} provided
a more thorough analysis of the most frequent errors made by the current SMT systems, the former
for English-French and the latter for English-Czech.

The error analysis was later used during developement of the rule-based APE Depfix\cite{depfix:2014},
which was designed to correct the errors made by the English-Czech SMT systems.
The system uses a set of finely hand crafterd rules that aim at identifying
and correcting morphological errors such as incorrect agreement or valency, which are
often encountered during English-Czech machine translation. It does not focus on a lexical errors
although some minor corrections, e.g. insertion of missing reflexive verbs, are made. The system
succeeded to improve output of various MT systems and was deployed as a stable part of the
Chimera\cite{biblio:BoRoChimera2013} MT system.

The idea behind the Depfix system seems promising, however, due to its rule-based nature, it is difficult
to apply the APE on different language pair since it would required costly modifications to the
existing set of rules. One of the goals of this thesis is to try and replace the rule-based blocks
by a statistical ones so they applied to the different target language MT output more easily simply
by training an appropriate statistical model.

\section{Thesis Structure}
%TODO: strucny popis depfixu? popis castych chyb?

%TODO: odkazy
In~chapter~\ref{chap:system_descr}, we are going to describe MLFix data processing
pipeline and introduce all the tools we use.
In~chapter~\ref{chap:data}, we will take a look at all datasets which we used during our system
developement and analyze their usefulness for our system.
In~chapter~\ref{chap:task_descr}, we are going to define the post-editing task, go through
various modifications of the task we considered and explain our though
process with support of the analysis of the available data.
In~chapter~\ref{chap:tuning}, we are going to describe the process of the developement
the statistical models for the MLFix post-editing component and present
a preliminary results of the evaluation of the trained models.
In chapter 6, we will evaluate the performance of the whole MLFix system
and analyze the level of contribution to the resulting performance
of each individual statistical compocomponent.
In chapter 7, we will briefly describe our modifications to the English-Czech pipeline
when used for the English-German language pair, then summarize the the differences
in the training data and model training between the two language pairs.
In chapter 8, we will present performance results of the modified English-German
pipeline.
We conclude our observations in chapter 9.\todo{Dodatek k attachmentum?}

%In chapter 2, we are going to describe all the datasets which we used
%either to analyze various the aspects of the post-editing process and
%to train our post-editing system.
%In chapter 3, we will define the post-editing task as a classification
%task and go through various scenarios, we considered. In this chapter
%we are also going to justify our decissions from the analysis of the
%available data.
%In chapter 4, we are going to describe the MLFix processing pipeline.
%In chapter 5, we will describe in detail our process of model training
%and feature selection.
%In chapter 6, we are going to present our experiments and results we
%achieved with the English-Czech language pair.
%After that in chapter 7 we are going to describe our modifications to the English-Czech
%pipeline so that it can be applied to the English-Czech MT output post-editing and
%in chapter 8 we will present results we acquired from experimenting with
%this pipeline.
%Finally in chapter 9 we will conclude the thesis.


\chapter{System Description}
\label{chap:system_descr}

In this chapter, we describe the main components
of MLFix system and take a closer look at the suggested processing pipeline.
We focus on English-Czech pipeline, however, we also describe modifications
needed to apply MLFix for other language pairs.

Full examples of currently deployed Treex scenarios (English-Czech and English-German)
are included in Attachment~\ref{attach:scen}.

%The main idea of MLFix system is to analyze the provided input data providing
%a set of input features 

\section{Processing Pipeline}

MLFix is, similarly to its predecessor, almost entirely implemented in the
Treex \citep{Popel:2010:TMN:1884371.1884406}\footurl{http://ufal.mff.cuni.cz/treex}
framework (formerly known as TectoMT).
The framework was originally created as a basis for English-Czech hybrid translation system, combining
rule-based modules with statistical models and using deep semantic language representation
for sentence translation. However, due to its modularity, it is
now used for various tasks of natural language processing (NLP) across different
languages. The framework was built to support the methodology of the theory of Functional Generative Description \citep{Sgall1967}
and was adapted to support sentence representation in Prague Dependency Treebank \citep{pdt20:2006}.
Mainly, it supports the representation of sentences on different layers of abstaction defined in FGD: morphological layer,
analytical layer and a tectogrammatical layer.\footnote{Usually referred to as m-layer, a-layer and t-layer where prefixes m-, a- and t- are also used to refer to the objects at the corresponding layer of abstraction.}
Because the tools currently available for tectogrammatical layer analysis are available for
a limited set of languages (mainly Czech and English, with others in progress),
we decided to use only the a-layer (surface syntax) for data representation.

\subsection{M-Layer Analysis}

MLFix analysis pipeline is derived from an existing Depfix pipeline
with a several modifications to make it easier to apply to different
target languages. MLFix takes a pairs of source sentences and their MT outputs, which
are aligned on a sentence level, as the input. Additionally, sentence-level aligned reference
translations are expected during the system training.
%The input data (source sentence + MT output aligned on a sentence level,
%or additionaly reference sentences for extracting the training data)
The input data are first read in parallel and stored into Treex internal representation.
Both source side and MT side are tokenized by a rule-based tokenizer, each token is then
represented by a separate m-node.

Next, lemmatization and part-of-speech (POS) tagging is performed.
For both English and Czech, we use MorphoDiTa \citep{strakova14:morphodita}\footurl{http://ufal.mff.cuni.cz/morphodita}
tool for morphological analysis and tagging. MorphoDiTa is also used for Czech lemmatization.
For English lemmatization, we use a rule-based block implemented in Treex.
It is important to provide a POS tagger for the target language that supports
relatively fine-grained morphological tags because our goal is to correct morphological
errors represented mainly through these tags.
It was reported by \citet[p. 33]{biblio:RoAutomaticpostediting2013} that the tagger produces significantly more
errors in morphological analysis of Czech SMT outputs in comparison with normal text. In Depfix, this is covered
by a rule-based block that identifies these errors and changes the incorrect morphological tag without
changing the surface form. We decided to omit this block and leave the issue to our statistical
component.

The last step we do in the scope of the m-layer analysis is a transformation of the morphological tags into
a more general representation. Optionally, we can apply a named-entity recognition
tool if one is available but it is not mandatory. For English, we use the Stanford
Named Entity Recognizer (NER) \citep{Finkel:2005:INI:1219840.1219885}\footurl{http://nlp.stanford.edu/software/CRF-NER.shtml},
for Czech, we use a simple rule-based NER. 

\subsection{Interset}

Since there are usually different tagsets used across individual languages,
often engineered for purposes of that specific language and with no standardized
tag representation,
we would be forced to modify our existing pipeline to some extent everytime
a new language would be introduced.
Therefore, we have decided to use Interset \citep{biblio:ZeReusableTagset2008}\footurl{https://ufal.mff.cuni.cz/interset}, an interlingua-based
representation of morphological tags from various tagsets. To be able to use this
representation to represent tags from a given tagset, a decoding/encoding
module is required. However, the support for various tagsets spanning
through different languages started growing lately mainly due to Universal Dependencies \citep{universal-dep:2016}\footurl{universaldependencies.org} project.

After the transformation, in following steps of the analysis, a choice
of a specific tagset becomes transparent because MLFix blocks only
have to deal with one well-defined set of features.

\subsection{Word Alignment}

In the next step, we create a word-level alignment for each sentence pair
using GIZA++ \citep{och:ney:2000}. We make one-to-one word alignment where possible
using the intersection symmetrization. This step helps us later with feature extraction
and with further processing of the target sentence.

In the process of training data extraction, we also create a simple alignment between
the MT sentence and the reference sentence exploiting forms, lemmas and tags
of the m-nodes. The alignment between the source sentence and the MT ouptut is then also
projected to the reference sentence.

\subsection{A-Layer Analysis}

After the m-layer analysis and the word alignment, we perform dependency parsing.
For English, we use Maximum spanning tree (MST) parser \citep{mcdonald:pereira:ribarov:hajic:2005}\footurl{http://sourceforge.net/projects/mstparser}
implemented in Treex framework. For SMT output, even though there might
be an existing dependency parser available for the target language, it is usually
trained on data that do not contain errors. Therefore, it has usually significantly
lower performance when applied on the SMT output.  research on Depfix has shown
that the knowledge of the dependency structure of the SMT output can provide additional
valuable information for identifying grammatical errors\footnote{Actually, in the case of Depfix
, the information about the dependency structure is crucial for most of the fixing components
because e.g. the parent-child relationship is examined almost everytime.}, thus improving
the Depfix performance.

For the time being, we have decided to build the dependency structures of the SMT output simply
by projecting the dependency structure of the source sentence onto the target side
using the word alignment we extracted in the preceeding step. The resulting structure
is likely to contain errors, but should be at least consistent thoughout our data.
To compensate for the lower accuracy of the extracted dependency structure we perform
dependency parsing of the reference sentences during training, if a proper parser
is available. For this purposes, we use the MST Parser for the Czech language as well.
We expect that the combination of dependency parser for the reference sentences and
a right choice of constraints applied when extracting training instances should
avoid polluting our training data with instances containing misleading context information
due to an incorrect structure of the SMT dependency tree.

For Czech SMT output, an implementation of the MST parser adapted for the SMT output
is already available \citep{biblio:RoDuUsingParallel2012}, however, so far we have not done any experiments
to determine whether the dependency structures provided by the adapted parser (which
should be more accurate than our projected trees) influence the final performance of our system.

%\subsection{Training pipeline}

%We also do preprocessing of data we use for training our statistical component.
%The pipeline is similar to the processing pipeline with addition of processing
%the reference sentences. The reference sentences are processed in a same way
%as the MT output, however they are not analyzed on the a-layer, only a simple
%lemma based word alignment with the MT output is made.

\section{Statistical Component}

After data preprocessing, we extract all available features and
apply a trained statistical model. The features are extracted separately for each node and passed
to a model. The model needs to accomplish these two goals:
\begin{enumerate}
    \item identify candidate words with incorrect morphology,
    \item fix incorrect morphological features.
\end{enumerate}
Of course, these two steps can be split between multiple separate components (and models).
We describe the post-editing process in more detail later in chapter~\ref{chap:tuning}.

We decided to use Scikit-Learn \citep{scikit-learn}\footurl{biblio:RoDuUsingParallel2012} toolkit to train and execute our models since
it has an easy-to-use and quite uniform interface, which allows us to try out different
machine learning (ML) methods simply by switching the model class. In Treex, we use a simple
wrapper to load and execute the trained Scikit-Learn model. If a support for a different ML implementation
is needed (e.g. VowpalWabbit\footurl{https://github.com/JohnLangford/vowpal\_wabbit/wiki}\textsuperscript{,}\footnote{At the time of writing this thesis, there is already
a limited support for VowpalWabbit in Treex.}), this wrapper can be easily modified to suit the requirements.

\section{Wordform Generation}

After we have identified morphologically incorrect words and assigned
them a new morphological categories, we need to generate new surface forms reflecting
the changes we have made.
This can be done either by rule-based component or with
the help of another statistical model. 

For Czech, we have used a morphological generator build upon the morphology of \citet{HajicHAB2004}
that is already part of the Treex framework. For other languages,
when there was not another option available,
we used Flect \citep{DBLP:conf/acl/DusekJ13}\footurl{https://ufal.mff.cuni.cz/flect}.
Flect is a language independent morphological generation tool also using Scikit-Learn
models. The tool learns morphological inflection patterns form an annotated corpus and
should be able to inflect even previously unseen words using lemma suffixes as features
and predicting the difference between the lemma and the necessary surface form given some
morphological specifications.

\section{Language Independence}

From the previous description, we can notice that MLFix still depends on several
language specific tools. It is quite dependent on the capability of a source
language analysis (in our case English), because we do not only need a POS tagger but also a dependency
parser. However, we assume that source sentences are usually grammatically correct
so it is much easier to provide required tools than their modified versions
targeted at the MT output.

We also require a specific decoder of the source and target POS tags into Interset feature
structures but Interset already covers a large variety of the most widespread tagsets available
and its support is still growing.

Finaly, aside from a language specific post-editing models (which we do not expect
to be reusable across different languages) we require a module that regenerates
the corrected wordform (either from the \samp{lemma+tag} or \samp{lemma+Interset features combination} specification).
A state-of-the-art tool might not be explicitly available for every language but if no
other option is provided, we can use a statistical form generator, in our case Flect.

\chapter{Available data}
\label{chap:data}

% dostupna data
% mnozstvi
% popis
% vyuziti

%XXX TOTO ZREJME AZ DO POPISU SYSTEMU
%In this chapter we are going to describe the format of the input data, that
%our system require for training and present all the available datasets.
%The differences between the suggested data can seem minor but they can
%have impact on the overall performance of the system.

%To train our system, we require sentence-level aligned set of source sentences,
%MT output and a reference sentences
%TODO: footnote - reference in this thesis as a triparallel data
%. The reference sentences can be either
%a classic reference translation of the source text or a result of a human
%post-editation of the MT output, they are however required because they
%provide use with the possible corrections of the MT output.
%Naturally, the closer the MT output is
%to the reference translation, the easier it should be for our system
%to extract valuable learning instances form the sentences. We also
%considered using only pair of sentence aligned pair of MT translated
%sentences and the reference sentences however, this way the trained
%model will lose some of the useful features which can be extracted from
%the triparallel data.
%XXX

In this chapter we are going to take a closer look at the available
sources of data and describe how they contributed to our research.

We came across a various sources of training data with various level
of usefulness. Data was usually available only in a smaller volume. Some of the sources
are:
\begin{itemize}
\item Khan's school human
post-edits of manually translated (EN$\rightarrow$CS) subtitles,
\item the Autodesk
triparallel data\footurl{https://autodesk.app.box.com/Autodesk-PostEditing},
\item log files of human post-editing done by Lingea for the
HimL\footurl{http://www.himl.eu/} project test dataset,

% uvest QT21?
%\item data from the QT21 project\footurl{http://www.qt21.eu/deliverables/annotations/},

%TODO: wmt - odkaz, footnote?
\item results from the previous workshops on machine translation (namely WMT10 dataset \citep{callisonburch-EtAl:2010:WMT},
and the datasets available for the upcoming WTM16).
\end{itemize}

In the following sections, we will describe each in more detail.

\subsection{Khan's school}

The data provided by Khan's school consist of English-Czech subtitles,
where the Czech part (usually manually translated from English) was manually
edited. During the analysis of the dataset,
we've noticed that most of the time, the corrections were made
mostly on a lexical level which is only natural since the Czech sentences
were created by a human translator.
Therefore we have concluded that this dataset has little to no value
for the task of training model for correcting errors in morphology.

%Therefore, we decided to avoid using this dataset for the time being or to
%rather treat the corpus as a simple bilingual data.

\subsection{Autodesk}

Autodesk data consist of English sentences which were machine translated into
a set of target languages (cs, de, pl etc.) complemented with human post-editing
of the MT output. However, these datasets are domain specific (mostly user documentations),
so they might not be very attractive to use with more general texts.
We weren't able to gather any information about the MT system that was used
to create the translated output. The biggest advantage of these data is
their larger volume when compared with other post-editing datasets so we
used them mainly for model developement and benchmarking of the used machine
learning methods.

\subsection{HimL-Lingea logs}

The data provided by Lingea\footurl{http://www.lingea.cz/} were collected when official test sets for
Healt In My Language\footurl{http://www.himl.eu/} (HimL) project were
created. The data consist mainly of the public health related texts.
The original English sentences were first
machine-translated to languages at which
the project was aimed (Czech, German, Polish and Romanian)
and then post-edited by professional
translators using Lingea's post-editing tool. The datasets are probably the most
detailed one since they consist of complete logfiles
describing elementary actions taken by human post-editors (such as selecting
phrases in a translated sentence, looking up alternative translations
in a dictonary etc.).

When we examined the data more closely,
we noticed that it is rather difficult to determine which actions are
useful for our machine learning process. Also, we were little disapointed
when we found out that most of the time, the post-editors preferred to
rewrite the whole sentence "from scratch"\footnote{By that, we mean that the
human post-editor usually preffered to rewrite the whole corrected sentence, even though
only a several changes (either lexical and morphological) were made, and delete the original.
This might have been also motivated by the need of reordering of the MT output.}
opposite to doing more atomic modifications to the provided MT output.

In the end, we decided to simply extract a triparallel data from these logs (the source
sentence + SMT output + result of the human post-editing). In the future,
we might consider to use other logged actions for model training.

\subsection{WMT datasets}

For the last decade, the workshop on machine translation (WMT) has aimed
to provide working grounds for many researchers in the field of machine
translation. It has been great source of the parallel data between English
on one side and various other languages on the other. Each year, the scope
of the workshop expands, including various new subtask related to machine translation,
such as several evaluation tasks and, more recently, automatic post-editing task.

The data available for the post-editing task usually contains a set of:
\begin{itemize}
    \item source English sentences,
    \item output of various MT systems, usually the ones that participate in the main
translation tasks,
    \item either a reference sentences or human post-editing of the MT output from the participating MT systems.
\end{itemize}

These datasets give us the opportunity to compare the performace of our post-editing
models when applied to the different systems. Even though the data available for the
WMT subtasks are often from various domains (news, IT, biomedical), the domain of the post-editing
data is more limited, mainly to the news articles.

\subsection{Other sources}

The sources listed above (each one to a different degree) can
be considered a knowledge base for examining the behaviour of a human post-editors as well
as training data for our system. We think that they provided us with
some interesting insight into the post-editor's thought process. On the other hand,
we have also considered using other sources since the data mentioned above are quite
limited.

One possible way to face the data sparsity is to use available parallel corpora.
These corpora (containing only \notion{source sentences + reference translations})
can be expanded by translating the source sentences and thus creating a set of 
sentences which contain MT generated errors and should be fixed to resemble the 
reference translation. These data can be then used to train post-editing models for that
specific SMT system\footnote{Obviously, the post-editing model training data have to be
different from the parallel data used for the training of the SMT system, so
some
jackknife sampling should be used with limited training data.} or possibly for
other SMT systems.
This method can surely help to overcome the aforementioned data acquisition bottleneck
since there is generally much more parallel data then post-edited sentences. For
English-Czech language pair, the natural choice of the parallel corpus would be
CzEng~1.0 \citep{czeng10:lrec2012}\footurl{http://ufal.mff.cuni.cz/czeng}.

%We didn't use this approach in the scope of this thesis however, it might
%be considered in our future work.
%\todo{posledni vetu az do shrnuti?}

Of course, this aproach introduces some additional noise
related to the post-editing task. For example, we can get fluent MT output which
is just a variation on the reference translation (possibly thanks to the different
choice of wording etc.). It can be therefore hard to distinguish these fairly correct training
instances from the incorrect ones.

The basic summary of the available data is shown in the table \Tref{avail-data}.

\subsection{Monolingual data}

We have also considered simply using bitext monolingual data (either \notion{MT output + post-edited sentences}
or \notion{MT output + reference translations}), however due to the nature of our processing
pipeline we would be much more limited when analyzing the training data. Also this way, we would
lose the additional information that can be extracted from the source sentences, which proved to be valuable in the practice.
\todo{citovat priklady?}

\begin{table*}[t]
\centering
\small

\begin{tabular}{lcccc}
\multirow{2}{*}{}  &  \multirow{2}{*}{\hash{} Sentences}  &  \multicolumn{3}{c}{\hash{} Tokens}  \\
&   & English & Czech (MT) & Czech (PE) \\
\hline
Khan's school & \tilda{}14k & \tilda{}93k & \tilda{}93k & \tilda{}93k \\
Autodesk & 46,916 & 490,005 & 456,697 & 441,645 \\
HimL-Lingea & 3892 & 60,142 & 51,428 & 56,485 \\
WMT10 & 2,489 & 54,021 & 44,578 & 45,422 \\
WMT16 & 2,999 & 57,418 & 48,037 & 48,915 \\
CzEng 1.0 & 15M & 206M & NA & 150M \\
\end{tabular}
\caption{Summary of the available post-editing data. Only English-Czech data is listed, however, for datasets
where data for other language pairs are available, their volume is roughly the same. We provide only rough estimates for the Khan's school data.
There is no information about the number of tokens in the MT part of CzEng because we have decide to abandon the
idea of creating a triparallel corpora for the time being.
}
\label{avail-data}
\end{table*}

%The
%number of parallel sentences and the number of tokens in the English source, the
%MT output (\equo{MT}) and the post-edited MT output (\equo{PE}). Only
%English-Czech data is listed since these datasets for other target languages
%(where available) are similar in volume.
%For Khan's school, we only
%provide estimates. The CzEng data set is not translated by any SMT at the moment,
%so the related information is omitted. We provide the information only for comparison.

%We do not include information about the QT21 data since we are yet to explore them.


\chapter{Task definition}

In this chapter we will present results of our closer inspection
of the available training data and describe the machine learning
task we assigned to the MLFix post-editing component.

\section{Feature extraction}

\section{Classifier definition}

\section{Oracle component}

\chapter{Model training}
\label{chap:tuning}

In this chapter, we are going to describe our process of developing
the statistical post-editing models. We will take a closer look at the feature selection
methods we have experimented with, the task of model selection and parameter
tuning and also the methods of evaluation we used during the tuning.
We only present experiment results for the separate statistical models,
results of the evaluation of the whole MLFix system is presented in the next chapter.

In this chapter, we will cover following two classification tasks: the identification
of the incorrect instances (words from the MT output with incorrect surface form)
and the prediction of the new morphological categories for the incorrect instances.

Our training process can be separated into three stages: in the first stage, we
focused on choosing a suitable machine learning method, we chose the most promising
one, in the second stage we tried to further increase the performance of the model
based on the chosen ML method by experimenting with various methods of feature filtering
and in the third stage, for each dataset, we search for the best hyperparameters
of the both the selected ML method and feature filtering method.
%We will take a closer look on both stages in the following subsections.


\section{Model evaluation methodology}

We have decided to do word-level evaluation during the model training, more precisely,
we have evaluated the performance of the models using the instances extracted by the
process we described in the previous chapter.

For both error detection and morphological category prediction task, we have defined
a baseline model for comparison. This model basically represents a predictor which
does not detect any errors (all instances are marked as correct) or keeps the original
morphological categories for each instance.

Aside from the standard accuracy metric, we have decided to measure precision and recall
of the trained models. We use standard definition of precision~(\ref{eq:prec-mod}) and recall~(\ref{eq:rec-mod}) using the following equations:
\begin{equation} \label{eq:prec-mod}
precision = \frac{TP}{TP + FP}
\end{equation}
\begin{equation} \label{eq:rec-mod}
recall = \frac{TP}{TP + FN}
\end{equation}
where definitions of \pojem{true positives} (TP), \pojem{false positives} (FP) and \pojem{false negatives} (FN)
vary slightly with each classification task. We also use f-measure to compare combined results of these metrics.

For error detection, each instance which was assigned same value as the \pojem{true prediction}
and different value than the \pojem{baseline} is marked as TP, instance
that was assigned value different both from the \pojem{true prediction} and \pojem{baseline}
is marked as FP and instance with \pojem{predicted value} equal to the \pojem{baseline}
but different from the \pojem{true prediction} is marked as FN.

For the morphological category prediction we use same definition of TP, however, the definitions
of FP and FN are altered in the following way (in both cases, we assume that the \pojem{true prediction}
does not match the \pojem{predicted value}):
\begin{itemize}
\item if the \pojem{baseline} value matches the \pojem{predicted value}, instance is marked as FN,
\item if the \pojem{baseline} value matches the \pojem{true prediction}, instance is marked as FP,
\item if the \pojem{baseline} does not match either one, the instance is marked as i\pojem{wrong positive} WP.
\end{itemize}

The WP is a special case which reflects the situation where predictor tries to predict new value (different
from the original one) but fails and returns just another incorrect value. We must also take into account
the fact, that the error detection classifier is a general model, which is not designed to distinguish
the types of morphological errors and the morphological predictor might be specialized only on a limited
set of the morphological categories. Therefore, sometimes we want it to just leave the \equo{incorrect}
instances unchanged because the reason it was marked as incorrect can be out of its scope.

In the end, we have found the basic accuracy metric as more informative for evaluating the morphology predictor,
not only because of the WP instances but also due to the nature of our training data. We use only instances
that are marked as incorrect by our heuristic. The resulting classifier then has to learn to predict the
morphological categories of the aligned reference nodes.

\section{Automatic error detection}

%data rep - # of features
%cv method
%baseline

We have found that not only during task specification but even during model training,
the task of identifying morphologically incorrect words in the text is more difficult
than the task of assigning the new morphological categories. We have faced several issues
during the training which we will describe in more detail in the following subsections.



\subsection{Unbalanced data problem}

In this task, we face the problem of the binary classification, where we decided
to assign value 0 to the instances that we consider correct and value 1 to the
instances that need to be corrected. The process of assigning these values to the
extracted training instances was described in the previous chapter. 

%For the model evaluation, we defined a baseline classifier, which assigned the 0 value
%to each instance, therefore marking all of the MT output as already correct.
In this task, the \pojem{baseline classifier} already achieved accuracy larger than 95\%
simply by marking all the instances as correct (class 0). However, we are more insterested
in maximizing the precision and recall of the class 1 predictions.
As we have already pointed out, this is caused by a fact that only a small portion of our
training instances is being marked as incorrect by our heuristic.
This became a severe issue since most of the machine learning methods rely more or less on accuracy during
the process of searching for the best hypothesis, however, it is the minority
class, that is our target during the error classification.

There are several methods that can be used to solve this problem\todo{citace?}: e.g. creating syntetic
training data by resampling instances with our minority class or removing some instances
belonging to the dominating class, weighting of the training instances or modification of the
cost function. The manipulation with the training data (upsampling, downsampling) is the easieast method,
however, the more it modifies the distribution of the classes over the training data, the worse can become the final
performance of the classifier on the real-life data. Still, the idea of modifying the distribution
of our training data is worth considering.

We have inspired ourselves by the work of \citet{2013_Jia_CoNLL_GrammaticalError} describing the classification of grammar errors in the texts
written by a human. They are aproaching the task, similarly to us, as an classification
problem. During training, they filter their training corpus using only around 5\% of
the sentences, because only those were the ones that contained grammatical errors.

In our case, if we look at the results produced by the Oracle classifier, we can see that at least
two-thirds of the MT sentences have not been modified. This means that at least two-thirds of the sentences
were considered morphologically \equo{correct}\footnote{This is very likely not to be true, but can at least assume
that they do not contain training instances marked as incorrect (as far as our heuristic goes).}
These sentences are still part of our original training data introducing a significant amount of training
instances marked as correct.
Therefore, to balance our data, we have decided to use only the training
instances from the sentences where at least one word was marked as incorrect. This way, we were
able to increase the portion of the minority class up to 10\%.

This less unbalanced dataset we have created can be used in two ways: we can simply treat
it like a downsampled data and use the trained models \equo{globally} (on every MT sentence),
or we can add another component that will be used to identify these \equo{incorrect} sentences
and than apply our error detection model. In the scope of this thesis we have chosen the former
approach.

%This obviously creates another task and that is detection of the \equo{incorrect} sentences.

%%% TOTO UZ JE ZMINENE NA ZACATKU %%%
%Our training process can be separated into two stages: in the first stage, we
%focused on choosing a suitable machine learning method, we chose the most promising
%one and in the second stage we tried to further increasing the performance of model
%based on the chosen method by experimenting with various methods of feature filtering.
%We will take a closer look on both stages in the following subsections.

\subsection{Machine learning method comparison}

There are many machine learning methods that support binary classification, so we
decided to only compare a limited subset of the available methods that are implemented
in the Scikit-Learn framework. For each classifier we tried several hyperparameter
settings to observe the changes in the classifier behaviour. However, we have made only
a rough examination of the hyperparameter configuration due to the number of observed
methods. In this stage, the goal was not to train the best possible classifier but
eliminate those, that are not suitable for the task.

We have measures the performance of the classifiers on several datasets:
WMT10 data, dataset extracted from the lingea logfiles (HimL) and
WMT16 newstest dataset translated by the Chimera system.
%For each datasets, only instances from the \equo{incorrect} (sentences contaning at least one error)
%sentences were used.
%one translated with the experimental neural network machine translation system (NMT)
%provided by the University of Edinburgh\todo{no ref, nezminovat?}.
Additionally, a counterpart was
created by substitution of the reference sentences with the Depfix output
to additionally measure the performance on the \equo{syntetic} data.
The summary of the size of the used training data is in~\Tref{wf-training-sum}

\begin{table*}[t]
\centering
\small

\begin{tabular}{lcc}
Dataset  &  \hash{} Instances  &  \hash{} Instances (filt.)  \\
\hline
WMT10  &  23,470  &  6,033  \\
HimL  & 7,234  &  2,469  \\
WMT16-chimera  &  26,942  &  7,047  \\
WMT10-Depfix  &  40,678  &  4,101  \\
HimL-Depfix  &  10,491  &  1,114  \\
WMT16-chimera-Depfix  42,021  &  1,897  \\
\end{tabular}
\caption{
    Summary of the size of the training data extracted from each dataset. We present
size before and after (filt.) removing the instances extracted from the \equo{correct} sentences.
}
\label{wf-training-sum}
\end{table*}


For the purposes of the coarse evaluation we have used each dataset separately for both training
and testing of the classifers, by performing one-against-the-rest 10-fold jackknife sampling.
Therefore the results presented in this stage should be considered only as an in-domain
performance for a specific MT system.

We have compared the following methods\todo{reference k metodam?}: logistic regression, ridge regression classifier,
random forests, extremely randomized trees and support vector machines (SVM) classifier.
In the~\Fref{wf-draft}, we can see the comparison of the performance of various classifiers based
on the F1-measure metric. We can see, that for the task of error identification, the support vector
machines with linear kernel might be the most suitable overperforming other methods in most of our
datasets. The \equo{baseline} peformance is not spectacullar, with score of less 0.3 for the normal data
and a slightly better score (\tilda{}0.5) for the Depfix data. Therefore we have decided to explore
further only SVM during the following stages of model training.

% with source
\begin{figure}
\centering
  \includegraphics[scale=0.7]{wf-class}
  \caption{
    Overview of the classifier performance (error detection).
We have tried several variations of the hyperparameters
for each classifier. The classifiers are ordered from the best to the worst. Only top 50 results
are shown for each dataset.
}
  \label{wf-draft}
\end{figure}

\subsection{Feature filtering}

During the model comparison, we have compared performance on two initial feature sets:
one did not contain any information about the source sentence (\tilda{}680 initial features)
and one with the source sentence features (\tilda{}1360 initial features). Before training
the features with zero variance were removed, however, no other feature selection has been
performed. We have noticed that the additional information provided by the source sentence features
significatnly improves performance of the majority of the ML methods. Therefore, we have decided
to use this feature set for the feature selection method comparison.

We have picked SVM as ML of our choice. We have picked several methods for feature
selection and compared their influence on the classifier performance. During the comparison
we used a SVM model with fixed hyperparameters. We compared the following methods for
feature selection: KBest selection (with chi-squared scoring function), selection of the percentile
of the features (based on the ANOVA F-test),
selection based on lasso regularization and selection through models with feature importance
scoring models (svm, random forest). As far as importance scoring goes, we compared different
model configurations and during features selection, only features with importance higher
than the mean of the feature importance distribution were selected.

The results of the feature selection method comparison are shown in~\Fref{wf-sel}. We can see
that most of the time the feature selection performed by either ANOVA percentile selector or
SVM slightly improved the model performance. We have therefore decided to use these methods during
the model training.

\begin{figure}
\centering
  \includegraphics[scale=0.7]{wf-sel}
  \caption{
    Overview of the performance of the SVM with linear kernel when combined with various feature selection methods.
We have tried several variations of the hyperparameters
for each method. The horizontal line marks the best performance without
a feature selection method. The methods are ordered from the best to the worst. Only top 50 results
are shown for each dataset.
}
  \label{wf-sel}
\end{figure}

%TODO: feature representation - vectorized dicts, binary features???
% chceme to zminovat?

\subsection{Model summary}

\todo{zhodnoceni natrehovanych modelu + comment}

\begin{table*}[t]
\centering
\small

\begin{tabular}{l|ccc}
Dataset  &  Precision  &  Recall  &  F1  \\
\hline
wmt10  &  0.45  &  0.11  &  0.17  \\
himl  &  0.40  &  0.16  &  0.23  \\
wmt16  &  0.77  &  0.21  &  0.33  \\
wmt10-depfix  &  0.25  &  0.20  &  0.22  \\
himl-depfix  &  0.27  &  0.24  &  0.26  \\
wmt16-depfix  &  0.75  &  0.38  &  0.50  \\
\end{tabular}
\caption{
    Final in-domain evaluation of the trained error detection models. The evaluation was performed
by a jack-knife one-vs-rest classification on each dataset. 
}
\label{wf-summary}
\end{table*}


\section{Prediction of new categories}

The second classification problem was predicting the correct morphological category for
the words that were marked as incorrect. Because we are using the Interset representation
of the morphological features we have several possibilities, how to handle this task such
as:
\begin{enumerate}
    \item predict each category separately,
    \item concatenate the features and treat them as a single prediction target,
    \item use the methods that support multitask classification,
\end{enumerate}

With the first option, the biggest issue is determining the order in which the classifiers
should be aplied. Additionally we have to decide if we also want to include the current node morphological features
into our model's feature set or use the newly predicted ones. The second approach eliminates this problem by predicting the
values simultaneously. This can, however, quite easily expand the set of the predicted values,
which usually leads to increase in data sparsity. This can be a big problem as we have
already shown that the amount of data available for the post-editing task (mainly the post-edited
data) can be quite low. The third option combines the first two by training an estimator
which handles multiple joint classification tasks, one for each morphological category. The
Scikit-Learn toolkit provides several classifiers which support this option.

Before jumping straight into training our classifier, it is important to examine what morphological
changes are made in our training data, how frequently they are made and how can they affect the resulting
surface form generation. For instance, we can predict new values of the \samp{punctype} category,
but it is very unlikely that it will affect the resulting wordform of any of the words
classified as incorrect, because this category is related strictly to punctuation. There are
of course less obvious examples and some categories, while being relevant for one target language
can be pointless for another.

For this reason we made a frequency analysis of the changes encountered
in our data, shown in the~\Fref{iset-barplot}. We can see that most of the time, only the grammatical
case was modified (more than 50\% of the instances for the Depfix-based datasets and more than 40\% of the instances for
the normal datasets). Other changes were a lot less frequent (less than 10\% of the modified instances).
We have also checked, the amount of modifications POS-wise. \Tref{changes-pos} summarizes the modification frequencies for each
POS. As a conclusion we have decided to focus on predicting these categories: grammatical case, number, gender
and animateness. These categories are relevant to the majority of the changed words, therefore, changes made by
a classifer predicting these categories should be significant.

\begin{table*}[t]
\centering
\small

\begin{tabular}{lc}
POS  &  Frequency  \\
\hline
noun    &   38\%  \\
adj     &   16\%  \\
adp     &   10\%  \\
verb    &   9\%  \\
adv     &   9\%  \\
\end{tabular}
\caption{
    Part-of-speech (POS) frequencies of the changed words.
}
\label{changes-pos}
\end{table*}


\subsection{Machine Learning method comparison}

We have decided to train four models: one predicting case only (C), one predicting case and number (CN),
 one predicting case, number and gender (CNG) and one predicting case, number, gender and animateness (CNGA).
We have used same datasets as in the previous task, however, this
time we have extracted only feature vectors of the instances that were marked as incorrect in our training data. Because
our predictors will only be classifying the incorrect instances, training them on the whole dataset would only create
unnecesary bias. On the other hand, this has made the training sets quite small (containing only a few hundreds of examples
at most). The summary of the training data is in~\Tref{cats-training-sum}. We have decided to train a separate classifier
for each dataset instead of combining the data together, because we can simply combine the models instead (e.g. via majority
vote, best prediction etc.). This allows us to evaluate a combined model on a test set of our choice simply by leaving
out the model which was trained on that specific test set.

\begin{table*}[t]
\centering
\small

\begin{tabular}{lc}
Dataset  &  \hash{} training instances  \\
\hline
WMT10  &  645  \\
HimL  & 338  \\
WMT16-chimera  &  722  \\
WMT10-Depfix  &  210  \\
HimL-Depfix  &  72  \\
WMT16-chimera-Depfix  &  99  \\
\end{tabular}
\caption{
    Summary of the size of the training data extracted from each dataset.
}
\label{cats-training-sum}
\end{table*}

Again, we had to decide which ML method should we use for this task. 
We have examined similar set of classifiers with similar
hyperparameters as in the error classification task to get a rough idea about their capabilities.
We have considered using the f-measure again, however, due to the nature of the training data (all
instances should be classified), the model accuracy metric seemed more informative.
The rough comparisons were made with the case classifier only.
Adding additional targets to the classifiers naturally lowers their overall accuracy,
however, their performance have been similar with respect to each other
The results are shown in~\Fref{cats-draft}. We can see, that even without any sophisticated parameter
tuning or feature filtering, the classifiers perform quite well. We can also notice that in most of the cases, the ensemble
methods (mainly extremely randomized trees) performed slightly better than the rest of the examined ML methods. Therefore, we have decided
to pick this method for further experiments.

\begin{figure}
\centering
  \includegraphics[scale=0.7]{iset}
  \caption{
    Frequency of the most changed Interset categories, grouped by a datasets. Categories containing
    "\textbar" symbol (e.g. gender\textbar{}number) represent changes made simultaneously.
}
  \label{iset-barplot}
\end{figure}

% with source
\begin{figure}
\centering
  \includegraphics[scale=0.7]{cats-class}
  \caption{
    Overview of the classifier performance (category prediction).
We have tried several variations of the hyperparameters
for each classifier. The classifiers are ordered from the best to the worst. Only top 50 results
are shown for each dataset.
}
  \label{cats-draft}
\end{figure}

\subsection{Feature selection}

We have decided to perform additional feature selection with models trained on the standard HimL dataset
and WMT10 dataset because there is still a reasonable room for an improvement. Again, we have tried two initial feature
sets, one using the source side features and one without them. We have not noticed any significant difference in performance
between models trained on these two initial feature sets so we decided to use the larger one and leave
the feature selection to the separate model. Additionally, due probably to the nature of the classifier, we have noticed
another slight improvement by adding the \pojem{source lemmas} (both of the aligned node and its parent) feature. Therefore, we have also
included them to the initial features.\footnote{We have not included the \pojem{MT lemma} feature, because we wanted to
try using models trained on Czech for German post-editing and this feature is too language-specific for that purpose.}

We have compared several methods of feature selection: KBest selection
(with chi-squared scoring function), selection based on lasso regularization and selection through
models (svm, random forest). Again, we have done a rough comparison of these methods, by trying out
several parameter configurations. We tested them on a model with a fixed parameters. The results
are in~\Fref{cats-sel}. We can see that regarding HimL dataset, the feature selection did not have
any positive influence on the resulting model. On the other hand, we have decided to use the SVM-based
feature selection for the WMT10 dataset model training.

\begin{figure}
\centering
  \includegraphics[scale=0.7]{cat-sel}
  \caption{
    Overview of the random forest performance when combined with various feature selection methods.
The methods are ordered from the best to the worst. The horizontal line marks the best performance without
a feature selection method. Only top 50 results are shown for each dataset.
}
  \label{cats-sel}
\end{figure}

%\subsection{Multitask models}
%todo???

\subsection{Model summary}

In the end, we have trained six different models, one for the each dataset presented in the rough comparison.
Aside from the case clasiffier, we have also compared performance of the chosen multitask classifiers:
the case-number (CN) and case-number-gender (CNG) classifier and the case-number-gender-animateness (CNGA)
classifier. These classifiers were trained using the same ML methods, each one was tuned separately.
The summary of the final in-domain in~\Tref{cats-summary}.

We can see that by including additional categories for the multitask classification the accuracy of the trained
model becomes naturally lower. However, we must take into account that we have only classified the predicted
values as correct or incorrect and we have not distinguished partially correct predictions. Brief overview
of the categories predicted by the CNGA classifer has shown us that usually a predictor misclassifies only
one or two of the target categories. Therefore, we have still used these models during the system evaluation.
Surprisingly, accuracy of the multitask models trained on WMT16 dataset dropped significantly when training a
more complex classifier, underperforming even the baseline predictor. It is possible that the chosen ML
method is not suitable for this dataset, even though it is performing well with the simple case classifier.

\begin{table*}[t]
\centering
\small

\begin{tabular}{l|cccc}
Dataset  &  Case(Base)  &  CN(Base)  & CNG(Base)  &  CNGA(Base)  \\
\hline
wmt10  &  72\%(35\%)  &  53\%(14\%)  &  40\%(7\%)  &  37\%(7\%)  \\
himl  &  69\%(34\%)  &  50\%(11\%)  & 41\%(5\%)  &  41\%(4\%)  \\
wmt16  &  93\%(50\%)  &  45\%(21\%)  &  34\%(10\%)  &  32\%(10\%)  \\
wmt10-depfix  &  92\%(45\%)  &  83\%(22\%)  &  69\%(19\%)  &  69\%(19\%)  \\
himl-depfix  &  92\%(47\%)  &  78\%(26\%)  &  73\%(23\%)  &  71\%(23\%)  \\
wmt16-depfix  &  93\%(54\%)  &  64\%(16\%)  &  56\%(11\%)  &  53\%(9\%)  \\
\end{tabular}
\caption{
    Final in-domain evaluation of the trained models. The evaluation was performed
by jack-knife one-vs-rest classification of the each dataset. Performance of the baseline
classifier is presented for comparison.
}
\label{cats-summary}
\end{table*}


%\todo{tohle az do final evaluace?}
%Since the domains of the datasets differ to a various degree, we have decided to combine the models trained
%on the different datasets, however, we do not combine different types of classifier. This combined model
%compares result of each separate classifier (with the prediction probability) and picks the most probable choice.


\chapter{System evaluation}
\label{chap:eval}

In this chapter, we present the results of the whole MLFix system evaluation.
We describe the datasets we have used during the evaluation and different
configurations of MLFix we have compared. Furthemore, we present evaluation
of the individual MLFix components.
We also present a comparison with the Depfix
system. We have performed both automatic and manual evaluation.

\section{Automatic evaluation}

During automatic evaluation, we have used mainly BLEU \citep{papineni:2002} translation
quality metric based on measuring the n-gram difference between the MT output
and the reference translation.
Even though it has its limitations, it is the most widely used
evaluation metric at the moment and it is considered a standard metric for automatic evaluation.
In addition, we have 
measured Translation Edit Rate (TER) \citep{Snover06astudy} because it measures
the quality of translation based on the amount of corrections needed to match
the reference translation. Lowering the amount of work needed to post-edit MT output
is one of the aims of MLFix system. Besides, the metric has been proved to provide
reasonable correlation with a human judgement.

We have evaluted MLFix on the following datasets: Autodesk, WMT10\linebreak
\citep{callisonburch-EtAl:2010:WMT}, WMT16 \citep{bojar-EtAl:2016:WMT1}, and HimL.
These datasets
were translated with various SMT systems.

Because the domains of the datasets differ to a various degree,
we have decided to combine the final models trained separately
on these datasets. We then combine these to receive a resulting prediction.
Naturally, during the evaluation, we combine the models in a leave-one-out
fashion. The models trained on the currently post-edited data are omitted.

%This combined model
%compares result of each separate classifier (with the prediction probability) and picks the most probable choice.
%We have used the combined model for the prediction of the new categories, for the error detection we have compared
%the combined model with the best model


\subsection{Error detection evaluation}

We have decided to use a combination of multiple binary classifiers in our error detection module.
However, we had to devise a voting scheme to interpret multiple outputs provided by these classifiers.
In the end, we have compared three basic methods: \pojem{Majority} vote, \pojem{AtLeastOne} method,
and \pojem{Average} prediction method.

The \pojem{Majority} vote basically chooses the class that was marked by a majority of classifiers.
We think that this way can help us compensate for the low precision of some of the classifiers we have trained.
The potential drawback can be combined low recall of the error detection module. We assume that
combining several low recall classifers can further bias the module toward the majority class when using
this method.

The \pojem{AtLeastOne} method classifies the word as incorrect if at least one classifier marks it
as incorrect. This method should counter the problem of combining several low recall classifiers to some
extend, however, we expect the precision to drop rapidly when adding more classifiers.

Because we are using classification models that support weighting of their predicted classes, we have
also considered \pojem{Average} voting scheme. If the combined classifier provide us not only with
a predicted class but also a some sort of confidence value (e.g. probability of the prediction correctness),
we can average these values and mark instance as incorrect if the averaged value exceed a certain threshold.
This method is similar to the \pojem{Majority} scheme with a difference that it can choose a result
predicted by a minority when their overall confidence in the prediction is large enough.

We show the comparison of these three methods in~\Tref{markonly-summary}. The values of the \equo{incorrect}
words were predicted by Oracle morphology predictor. We can notice that the AtLeastOne method achieved
significantly better BLEU score than the other two. This was quite expected and in this case it does
not necessarily mean that the resulting sentence improved from the human evaluation standpoint, because
the situation is similar to the heuristic selection we presented in the task definition. Even though
many wordforms were changed by this method to reflect the wordforms in the reference sentence, it
might have lowered the overall fluency of the translated text. On the other hand, the BLEU scores were
still under the Oracle classifier threshold, therefore, the method seems promising.

As for the other two methods, the Average
voting method seems to perform slightly better than the Majority voting but the difference is not significant.

\begin{table*}[t]
\centering
\small
\resizebox{0.9\textwidth}{!}{

\begin{tabular}{|l|l||c||c||c|c|c|}
\hline
Dataset  &  System  &  Oracle  &  Base  &  Majority  &  AtLeastOne  &  Average  \\
\hline
\hline
Autodesk  &  NA  &  49.20  &  47.82  &  47.89 (+0.06)   &  \bf{48.47} (+0.65) &  47.89 (+0.06)  \\
\hline
HimL  &  Moses  &  23.33  &  20.66  &  20.69 (+0.02)  &  \bf{22.08} (+1.41)  &  20.69 (+0.02)  \\
\hline
WMT10  &  CU Bojar  &  16.70  &  15.66  &  15.84 (+0.18)  &  \bf{16.30} (+0.64)  &  15.77 (+011)  \\
\hline
\multirow{2}{*}{WMT16}  &  UEDIN NMT  &  27.31  &  26.31  &  26.31 (0)  &  \bf{26.49} (+0.18)  &  26.31 (0)  \\
&  CU Chimera  &  23.13  &  21.72  &  21.79 (+0.07)  &  \bf{22.01} (+0.28)  &  21.79 (+0.07)  \\
\hline
\hline
Autodek-D  &  NA  &  98.32  &  96.93  &  97.99 (+1.05)  &  \bf{97.99} (+1.05)  &  98.31 (+1.37)  \\
\hline
HimL-D  &  Moses  &  96.05  &  94.23  &  94.62 (+0.39)  &  \bf{96.04} (+1.81)  &  94.62 (+0.39)  \\
\hline
WMT10-D  &  CU Bojar  &  95.72  &  93.42  &  94.76 (+1.33)  &  \bf{95.61} (+2.18)  &  94.80 (+1.37)  \\
\hline
WMT16-D  &  CU Chimera  &  97.51  &  96.56  &  97.04 (+0.48)  &  \bf{97.47} (+0.9)  &  97.05 (+0.49)  \\
\hline
\end{tabular}

}
\caption{
    Automatic evaluation of the error detection module using different voting methods to interpret
output of multiple models using the BLEU score (mutliplied by 100).
For comparison, evaluation of Oracle classifier is also provided.
Values in brackets indicate the difference between the method and the baseline (Base) MT output.
Datasets with the \pojem{-D} suffix have Depfix output in place of reference sentences.
}
\label{markonly-summary}
\end{table*}


\subsection{Category prediction evaluation}

\begin{table*}[t]
\centering
\small
\resizebox{0.9\textwidth}{!}{

\begin{tabular}{|l|l||c||c|c|c|c|c|c|}
\hline
Dataset  &  System  &  Oracle  &  Case  &  CN  & CNG  &  CNGA  & Comb  &  Comb-D  \\
\hline
\hline
Autodesk  &  NA  &  49.20  &  47.89  &  47.91  &  \bf{48.22}  &  48.21  &  47.90  &  47.90  \\
\hline
HimL  &  CU Chimera &  23.33  &  21.14  &  20.97  &  21.36  &  \bf{21.49}  &  21.03  &  21.08  \\
\hline
WMT10  &  CU Bojar  &  16.70  &  15.84  &  15.82  &  \bf{15.95}  &  \bf{15.95}  &  15.82  &  15.82  \\
\hline
\multirow{2}{*}{WMT16}  &  UEDIN NMT  &  27.31  &  26.43  &  26.48  &  26.47  &  \bf{26.50}  &  26.44  &  26.43  \\
&  CU Chimera  &  23.13  &  21.94  &  21.99  &  22.02  &  \bf{22.05}  &  22.02  &  21.94  \\
\hline
\hline
Autodek-D  &  NA  &  98.32  &  98.12  &  98.09  &  \bf{98.17}  &  98.16  &  98.08  &  98.10  \\
\hline
HimL-D  &  CU Chimera  &  96.05  &  95.23  &  95.04  &  95.41  &  \bf{95.42}  &  94.98  &  95.20  \\
\hline
WMT10-D  &  CU Bojar  &  95.72  &  95.03  &  94.99  &  \bf{95.10}  &  95.07  &  94.96  &  95.02  \\
\hline
WMT16-D  &  CU Chimera  &  97.51  &  97.22  &  97.23  &  97.26  &  \bf{97.27}  &  97.20  &  97.24  \\
\hline
\end{tabular}

}
\caption{
    Automatic evaluation of the morphological category prediction module. For comparison, evaluation of the
Oracle classifier is also provided. Datasets with the \pojem{-D} suffix have Depfix output in place of reference sentences.
}
\label{fixonly-summary}
\end{table*}

\section{Manual evaluation}

For manual evaluation, we have used the HimL dataset. We have used two independent annotators, both were presented with 
with a set of sentences randomly selected from the dataset.

A total of \todo{cislo} sentences were evaluated by the annotators. The results are shown in\todo{table}.
%okec

%okec k agreementu

\chapter{English-German}
\label{chap:german}

In this chapter, we describe changes we made to the English-Czech MLFix pipeline
to be able to apply the system to the English-German
SMT outputs. We summarize the data available for the model training
and evaluate the system in a similar way we did with the English-Czech
pipeline.

\section{Processing Pipeline Modifications}

As we already pointed out, we focused on making the processing pipeline as independent
on the target language as possible. However, we still had to replace some of the tools
used during the Czech analysis to be able to correctly process German sentences.

Again, we used the Treex framework as a backbone of the processing pipeline and
necessary 3rd party tools were implemented into the framework via wrappers.
After the sentences are read in parallel, they are processed separately, English sentences
following the same scenario as in the English-Czech pipeline.
German is tokenized, this time by a set of regex rules inspired by the Tiger corpus \citep{Brants2004}
with main focus on abbreviations, ordinal numbers and compounds connected by hyphens.

Next, lemmatization and morphological POS tagging is performed by a Mate tools\footurl{http://www.ims.uni-stuttgart.de/forschung/ressourcen/werkzeuge/matetools.en.html}
toolkit. The tagger is using CoNLL2009 \citep{CoNLL-2009-ST} tagset. We also considered using
Stanford POS Tagger \citep{Toutanova:2000:EKS:1117794.1117802}\footurl{http://nlp.stanford.edu/software/tagger.shtml},
however, the tagset it uses contains only coarse tags with little morphological information.
To convert the CoNLL2009 tags to Interset, we use a decoder which was already available at the
time of our research.

For the word alignment, we use GIZA++ again. Similar to English-Czech, we produce one-to-one word
alignment between the source sentences and the MT output
via intersection symmetrization. The alignment model was trained on the European
Parliament Parallel corpus \citep{koehn2005epc}\footnote{http://www.statmt.org/europarl/} (Europarl)
containing nearly two million sentences. 
During training, we create the MT-REF and SRC-REF alignments in a same fashion we did in the original
pipeline.

The dependency structure of the MT output is created by projecting the English dependency
structures on the MT sentences. When we process the reference sentences during training, we 
use a graph-based parser implementation \citep{Bohnet:2010:VHA:1873781.1873792} which is also a part of the
Mate tools toolkit. We stop the analysis at the a-layer, but again, further analysis
of the sentences at the t-layer to gain additional features for extraction might be helpful in the future.

We reused the statistical component used in the English-Czech pipeline, because it was designed to be
language independent (with exception of the statistical models). For wordform generation we use
Flect morphology generation tool mentioned earlier. We trained the generator on a small
fraction (around one hundred thousand sentences) of the Europarl corpus. The tool is trained on a set
of features based on a combination of lemma$+$Interset producing the inflected word. The feature set
was copied from the Dutch feature set so it might not contain all the useful features. The accuracy
of the inflection model measured on a separate test set was around 94.5\%. However, when we briefly examined
the sentences produced via Oracle, we noticed that much larger amount of words was flected incorrectly.

It is not in the scope of this thesis but it is a future goal to replace the current inflection model
with a better solution, either stochastic or rule-based.

\section{Data Analysis}

We were able to collect only a smaller variety of data for English-German compared to English-Czech
 mostly due to some datasets we mentioned earlier simply not being available for this language pair. Still, we were 
able to gather the following datasets: WMT16, HimL and Autodesk. Note that in case of Autodesk dataset,
the size of English-German corpus is about three times bigger than the size of English-Czech (around 120k sentences).
For this reason, we decided to include Autodesk dataset into our training data even though it covers
a quite specific domain.

When extracting the training instances for the model training we followed same scenario
as before using the same heuristic (WrongForm3) to identify \equo{incorrect} wordforms. We also used the Oracle
classifier to gather information about possible improvements this heuristic can bring when applied to German.

We also performed quick manual evaluation of the output produced by the Oracle classifier by a non-native
German speaker.
The evaluation was performed on the HimL dataset.
%this time because we have thought that medical domain of HimL dataset
%would be too difficult for a non-native speaker to evaluate.
Due to the limited
resources we used only a single annotator for this evaluation task. The evaluator, not being familiar
with the MLFix system, was presented with a set of instances containing the following: randomly shuffled MT output and Oracle output,
the source English sentence and the German reference translation. The results of the evaluation are shown in~\Tref{oracle_de-maneval}.
We decided to only correct morphological categories by Oracle leaving the surface form generation to Flect module because
we wanted to see the best possible outcome that can be achieved by the statistical fixing components. In the future
the Oracle evaluation with \equo{Oracle} surface form generation might also be a valuable source of information.
This choice resulted in worse performance of the Oracle classifier when compared to Czech Oracle.
Therefore,
we still chose to use the same heuristic for marking incorrect instances in the training data, because
it probably was not because of the mark method that the Oracle performance dropped.

\begin{table*}[t]
\centering
\small

\begin{tabular}{l|cc|ccc|cc}
Reference  &  Evaluated  &  Changed  &  $+$  &  $-$  &  0  &  Precision  &  Impact  \\
\hline
Post-edits  &  800  &  77  &  20  &  39  &  18  &  33.9\%  &  2.5\%  \\
\end{tabular}
\caption[Manual evaluation of the German Oracle classifier]{
Results of the manual evaluation of the ideal fixing module based on the same heuristic
that was used in the English-Czech pipeline (WrongFrom3). Sentences were taken from HimL dataset. For wordform generation, a statistical component
was used impacting the overall performance.
}
\label{oracle_de-maneval}
\end{table*}

After the evaluation of the data extraction method we analyzed the extracted instances and compared them
to the information we gathered during Czech data analysis. \Fref{iset_de-barplot} shows the frequency of the changed Interset
categories. Again, case was the most changed category among our data followed by gender and number, either as a standalone change or
as a part of a clustered change. This led us to training the morphological prediction models in a same manner as in the English-Czech
pipeline dropping the animateness (\pojem{CNGA}) models this time. To make sure that we can make similar assumptions during
the model development we also checked the distribution of changes made per individual POS classes. The summary is shown in~\Tref{changes_de-pos}.
Even though the distribution is slightly different, nouns and adjectives are still the most changed words which further supports
our decision for training Case, CN and CNG models.

\begin{figure}
\centering
  \includegraphics[scale=0.7]{iset_de}
  \caption[Change frequency of German morphological categories]{
    Frequency of the most changed Interset categories in German data, grouped by a datasets. Categories containing
    "\textbar" symbol (e.g. gender\textbar{}number) represent changes made simultaneously.
}
  \label{iset_de-barplot}
\end{figure}

\begin{table*}[t]
\centering
\small

\begin{tabular}{lc}
POS  &  Frequency  \\
\hline
noun    &   35\%  \\
adj     &   25\%  \\
punc	&	10\%  \\
adp     &   8\%  \\
conj    &   7\%  \\
\end{tabular}
\caption{
	Change frequency of various POS classes in German.
%    Part-of-speech (POS) frequencies of the changed words. Only the 5 most
%	frequent classes are displayed.
}
\label{changes_de-pos}
\end{table*}


\section{Model Development}

We skipped the process of choosing proper ML and feature selection method
assuming that the ones used for training models for English-Czech pipeline will
also be sufficient for German. We only searched for the best combination
of hyperparameters during the model development process. We used both source
and MT features with addition of the source language lemmas to create the initial feature set.
We present a summary
of the available training data for both error detection and morphological prediction in~\Tref{wf-cat-data-sum}.
Again, we can see, that the training data are quite small with exception of the Autodesk
dataset. That is also the main motivation behind including it in our training data.

\begin{table*}[t]
\centering
\small

\begin{tabular}{ll|ccc}
Dataset  &  System &  \hash{} Instances  &  \hash{} Instances (filt.)  &  \hash{} Incorrect  \\
\hline
HimL  &  Moses  & 12,067  &  2,729  &  369  \\
WMT16  &  UEDIN-NMT  &  22,353  &  3,340  &  344  \\
WMT16  &  UEDIN-PBMT  &  21,394  &   3,727  &  412  \\
Autodesk  &  $-$  &  1,263,750  &  124,698  &  17,268  \\
\end{tabular}
\caption[Summary of the extracted German training data]{
    Summary of the size of the training data extracted from each dataset. We present
size before and after (filt.) removing the instances extracted from the \equo{correct} sentences.
The size of datasets for category predictor training is presented in the \pojem{Incorrect} column.
}
\label{wf-cat-data-sum}
\end{table*}

The summary of the final error detection models is in~\Tref{wf_de-summary}. Surprisingly,
while having quite similar precision to the Czech models they achieved much better
recall overall. At this moment, we cannot say if it is caused by a ML method choice or a suitable
initial feature set, however, since there is still room for improvement, we consider investigating
the issue further in the future.

\begin{table*}[t]
\centering
\small

\begin{tabular}{ll|ccc}
Dataset  &  System  &  Precision  &  Recall  &  F1  \\
\hline
HimL  &  Moses  &  0.39  &  0.56  &  0.46  \\
WMT16 &  UEDIN-NMT  &  0.28  &  0.43  &  0.34  \\
WMT16 &  UEDIN-NMT  &  0.29  &  0.50  &  0.37  \\
Autodesk  &  $-$  &  0.41  &  0.77  &  0.53  \\
\end{tabular}
\caption[Model summary (German) - error detection]{
    Summary of the in-domain performance of the final German error detection models.
}
\label{wf_de-summary}
\end{table*}

In~\Tref{cats_de-summary}, we present performance of the final German
mophology prediction models. The results are not very different from the Czech models with
HimL model being slightly better in this case, possibly due to a lesser amount of possible values
of the case category. Surprisingly, the accuracy of  a model trained on the Autodesk
dataset drops only a little with increasing complexity suggesting that increasing
the amount of training data can significantly improve the overall accuracy of the
resulting model.

\begin{table*}[t]
\centering
\small

\begin{tabular}{ll|ccc}
Dataset  &  System  &  Case(Base)  &  CN(Base)  & CNG(Base)  \\
\hline
HimL  &  Moses  &  84\%(29\%)  &  79\%(13\%)  &  70\%(5\%)  \\
WMT16  &  UEDIN-NMT  &  57\%(46\%)  &  48\%(27\%)  &  34\%(22\%)  \\
WMT16  &  UEDIN-PBMT  &  57\%(38\%)  &  47\%(21\%)  &  35\%(18\%)\\
Autodesk  &  $-$  &  96\%(28\%)  &  95\%(17\%)  &  93\%(8\%)  \\

\end{tabular}
\caption[Model summary (German) - morphological prediction]{
    Summary of the in-domain performance of the final German category prediction models
	and its comparison with the baseline predictor.
}
\label{cats_de-summary}
\end{table*}

\section{Evaluation}

We followed the same procedure during German MLFix evaluation as for
Czech. We compared several configurations of the morphological prediction
module and error detection module separately first, using Oracle to substitute the other
module, then we evaluated the whole MLFix system. We compared similar configurations
we used in Czech MLFix: \pojem{Case}, \pojem{CN}, \pojem{CNG} and \pojem{Combined} model configuration for morphological
prediction and \pojem{Majority}, \pojem{AtLeastOne} and \pojem{Average} voting scheme for error
detection. For the whole system evaluation, we picked the two most promising configurations
and compared their performance. We also measured a performance of the best Czech configuration
(using Czech models) when applied to German. We used BLEU scoring metric during automatic evaluation.
The evaluation was performed on the following datasets: Autodesk, HimL Lingea logs and WMT16.
For each evaluated dataset, the models trained on the corresponding dataset were excluded.

We present the results of morphological prediction module evaluation in~\Tref{fixonly_de-summary}.
Once again, CNG configuration proved to be the reliable choice, having the best score on each dataset.
However, the overall improvement in BLEU score is much lower when compared with Czech version
of MLFix. This is little surprising because the individual performance of the morphological prediction
models measured during training was fairly equal and in some cases even better than the one of the Czech models.
It is possible that this might be either a result of lower diversity in our data (most of the data belongs
to Autodesk dataset) or a consequence of the poor performance of the inflection module.

\begin{table*}[t]
\centering
\small
\resizebox{0.98\textwidth}{!}{

\begin{tabular}{|l|l||c||c||c|c|c|c|c|c|}
\hline
Dataset  &  System  &  Oracle  &  Base  &  Case  &  CN  & CNG  & Comb  \\
\hline
\hline
Autodesk  &  $-$  &  46.23  &  45.90  &  45.96 (+0.06)  &  45.95 (+0.05)  &  \bf{46.02} (+0.12)  &  45.98 (+0.08)  \\
\hline
HimL  &  Moses &  31.94  &  30.95  &  31.37 (+0.41)  &  31.29 (+0.34)  &  \bf{31.59} (+0.63)  &  31.46 (+0.50)\\
\hline
\multirow{2}{*}{WMT16}  &  UEDIN NMT  &  35.05  &  34.82  &  34.82 (0)  &  34.82 (0)  &  34.82 (0)  &  34.82 (0) \\
&  UEDIN PBMT  &  29.38  &  29.11  &  29.11 (0)  &  29.11 (0)  &  29.11 (0)  &  29.11 (0)  \\
\hline
\end{tabular}

}
\caption[Automatic evaluation of the German morphological prediction module]{
    Automatic evaluation of German morphological prediction module using BLEU metric
	and the relative improvement over the baseline MT output.
	Values are multiplied by 100 for easier reading.
	Performance of
	Oracle classifier is provided for comparison.
	The best model for each dataset is printed in bold.
}
\label{fixonly_de-summary}
\end{table*}


In~\Tref{markonly_de-summary}, we present the results of the evaluation of the error detection module.
We can see that the module was performing quite poorly, not bringing any improvement to any dataset
at all. Due to the nature of the error detection module evaluation (new morphological categories are taken
from the reference sentences and the wordform is regenerated by the inflection module),
we suspect that the main reason behind the poor performance is truly the
inflection module. Aside from that, we can see that this time it was the Majority scheme which performed
 much better than the rest. However, the results might only point to the fact that the Majority scheme
 marked the smallest number of instances as incorrect thus worsening the MT output much less than the
 other two.

\begin{table*}[t]
\centering
\small
\resizebox{0.98\textwidth}{!}{

\begin{tabular}{|l|l||c||c||c|c|c|c|c|c|}
\hline
Dataset  &  System  &  Oracle  &  Base  &  Majority  &  AtLeastOne  & Average  \\
\hline
\hline
Autodesk  &  $-$  &  46.23  &  45.90  &  \bf{45.80} (-0.10)  &  45.52 (-0.38)  &  45.79 (-0.11)  \\
\hline
HimL  &  Moses &  31.94  &  30.95  &  \bf{30.89} (-0.05)  &  30.17 (-0.77)  &  30.58 (-0.36)  \\
\hline
\multirow{2}{*}{WMT16}  &  UEDIN NMT  &  35.05  &  34.82  &  \bf{33.25} (-1.56)  &  30.15 (-4.67)  &  30.78 (-4.03)  \\
&  UEDIN PBMT  &  29.38  &  29.11  &  \bf{27.96} (-1.15)  &  25.41 (-3.7)  &  25.97 (-3.14)  \\
\hline
\end{tabular}

}
\caption[Automatic evaluation of the German error detection module]{
    Automatic evaluation of the error detection module using different voting methods to interpret output
of multiple models using BLEU metric. Values are multiplied by 100 for easier reading,
and the relative improvement over the baseline MT output.
Performance of the
Oracle classifier is provided for comparison.
The best model for each dataset is printed in bold.
}
\label{markonly_de-summary}
\end{table*}

Nevertheless, we decided to pick \pojem{Majority-CNG} (\pojem{Major-CNG}) and \pojem{Average-CNG} (\pojem{Avg-CNG}) configurations
for the final evaluation. Their performance is summarized in~\Tref{final_de-summary}. Again, both
systems performed poorly and it might look like the Czech MLFix provided the best results. Therefore,
we also provide a summary of the number of sentences that were changed by each system in~\Tref{final_de-chgd}.
We can see that our suspicion that the negative score correlates with the recall of each configuration was not
completely wrong. When we compare results we gathered during the Oracle evaluation and model
development for each language, we do not think that the poor performance during final evaluation was
caused mainly by the fixing
components. We suspect that the current performance bottleneck lies within the inflection module.
A more thorough investigation of the inflection module is required in the future, with a possible
replacement with an alternative.

\begin{table*}[t]
\centering
\small
\resizebox{0.98\textwidth}{!}{

\begin{tabular}{|l|l||c||c|c|c|}
\hline
Dataset  &  System  &  Base  &  Major-CNG  &  Avg-CNG  &  CS-Best  \\
\hline
\hline
Autodesk  &  $-$  &  45.90  &  45.64 (-0.26)  &  45.54 (-0.36)  &  45.82 (-0.08)  \\
\hline
HimL  &  Moses &  30.95  &  34.81 (-0.56)  &  28.35 (-2.60)  &  30.95 (0) \\
\hline
\multirow{2}{*}{WMT16}  &  UEDIN NMT  &  34.82  &  30.46 (-4.36)  &  19.95 (-14.87)  &  34.81 (0)  \\
&  UEDIN PBMT  &  29.11  &  25.70 (-3.41)  &  17.02 (-12.09)  &  29.11 (0)  \\
\hline
\end{tabular}

}
\caption[Final German MLFix evaluation]{
    Final evaluation of the Englsh-German configuration of MLFix using BLUE metric.
	Values are multiplied by 100 for easier reading. Majority-CNG and Avg-CNG methods were compared
with the best English-Czech configuration.
}
\label{final_de-summary}
\end{table*}

We also performed manual evaluation of the Avg-CNG configuration\footnote{At the moment, we cannot surely tell what is
the best possible configuration for German, so we simply followed the approach from English-Czech pipeline.}. Two
independent non-native German speakers jointly evaluated 444 changed sentences. The results of manual evaluation are in~\Tref{maneval_de-final}.
They both evaluated a subset of 141 to measure their inter-annotator agreement, shown in~\Tref{maneval_de-agree}.
We can see that the impact of the German pipeline was similar to the Czech pipeline. However, the low precision ($\sim$13\%)
confirms the results measured by the automatic metric. This is further supported by a reasonably high inter-annotator agreement of 83\%.

\begin{table*}[t]
\centering
\small

\begin{tabular}{|l|l||c|c|c||c|}
\hline
Dataset  &  System  &  Major-CNG  &  Avg-CNG  &  CS-Best  &  Sent.  \\
\hline
\hline
Autodesk  &  $-$  &  6,626  &  8,794  &  2,535  &  124,498  \\
\hline
HimL  &  Moses  &  145  &  421  &  0  &  800  \\
\hline
\multirow{2}{*}{WMT16}  &  UEDIN NMT  &  2,078  &  2,949  &  3  &  2,999  \\
&  UEDIN PBMT  &  2,014  &  2,921  &  5  &  2,999  \\
\hline
\end{tabular}

\caption[Final German MLFix evaluation - number of changed sentences]{
    Number of sentences changed by different systems. Total number of
sentences in each dataset (\pojem{Sent.}) is listed for reference.
}   
\label{final_de-chgd}
\end{table*}



\begin{table*}[t]
\centering
\small

\begin{tabular}{l|cc|ccc|cc}
  &  Evaluated  &  Changed  &  $+$  &  $-$  &  0  &  Precision  &  Impact  \\
\hline
A  &  640  &  313  &  36  &  263  &  14  &  12.0\%  &  5.6\%  \\
B  &  320  &  141  &  18   &  118  &  5  &  13\%  &  5.6\%  \\ 
\hline
Total &  960  &  444  &  54  &  381  &  19  &  12.4\%  &  5.6\%  \\
\end{tabular}
\caption[German MLFix manual evaluation]{
Results of the manual evaluation of chosen German MLFix configuration (Avg-CNG)
on a subset of HimL dataset.
}
\label{maneval_de-final}
\end{table*}

\begin{table*}[t]
\centering
\small

\begin{tabular}{c|cc|c}
 A/B  &  $+$  &  $-$  &  0  \\
\hline
$+$  &  7  &  10  &  1  \\
$-$  &  6  &  109  &  3  \\
\hline
0 &  0  &  2  &  3  \\
\end{tabular}
\caption[German MLFix manual evaluation - inter-annotator agreement]{
    Matrix containing inter-annotator agreement of German MLFix manual evaluation.
}
\label{maneval_de-agree}
\end{table*}
 


\chapter{Conclusion}
\label{chap:conclusion}
%\chapter*{Conclusion}
%\addcontentsline{toc}{chapter}{Conclusion}

In this thesis, we presented MLFix, an automatic post-editing tool focusing on statistical
post-editing of incorrect morphology in machine translation output. The system was
developed as a successor of a rule-based system, Depfix, with the aim to generalize some of its
rules to a stochastic model which can be applied across languages.

During the development, we had to find a compromise between the level of language independence
and overall usefulness of the system. In the end, we have chosen a unique approach to the problem of correcting
the morphology by solving a two-step classification task: error detection and morphological prediction.
We have faced a problem of automatic identification of correct/incorrect training instances
which we have solved with a fairly effective heuristic. Still, further refinement of the
training data extraction method is desired in the future.

Out of the two classification tasks, the morphological prediction proved to be much easier.
The resulting models, while being very good at predicting simple categories (e.g. morphological case)
manifested much lower individual performance with increasing task complexity. However when
combined together, they performed quite well.
Also, since we have used only really small datasets for model training, we think that these models
can be improved in the future by increasing the training data or providing additional features.
This claim is supported by the model trained on a much larger Autodesk dataset, which achieved really good
in-domain performance even when it was trained to classify multitask problem (prediction of case-gender-number)
Furthermore, we think that
these models have a potential use even in different fields of application, e.g. as a part of automatic
correction suggestion in a human post-editing framework.

The task detecting targets for our morphological prediction tool became main a hurdle
during the development of MLFix. Aside from the correct identification of the training instances
in our data, there was also an issue with highly unbalanced training set which we partially resolved
by upsampling the minority class and filtering out instances from the \equo{correct} sentences.
Even though the resulting models' performance seemed unsatisfactory at first, they performed resonably
well during final evaluation as far as precision of the resulting system was concerned. The weaker
side was the fairly low overall impact on the MT output. As far as future improvement goes, the results
achieved during model training on the large Autodesk dataset suggest that the performance can still be
improved simply by increasing the amount of our training data.

As we are mentioning using larger training datasets in the future, in the scope of this thesis we have
focused mainly on investing human post-edited data that are, at the moment, available in much smaller
volumes than data containing reference translation. However, we have tried training few models on smaller
datasets with reference sentences instead of human post-editing. The resulting models still performed
fairly well if the reference sentences were reasonably similar to the MT output. Lastly, we have
also examined data created by replacing the human post-editing with Depfix output resulting in reliable
source of training data. These data tend to be much more sparse (as a result of Depfix
impact on the MT output) and the method is currently restricted to English-Czech language pair only.
However, if we can achieve adapting Czech models for other languages in the future, this method might
become viable.

During the final evaluation, the system performed well when measured with the BLEU scoring metric.
These results were confirmed to some extent by manual evaluation, however, we are still not
completely confident in the results and suggest evaluating MLFix on much larger scale in the future.
Surprisingly, MLFix was able to surpass Depfix when it was applied to the output of NMT system.
This result caught our attention and will be investigated closely in the near future. If confirmed,
the application to the increasingly popular approach to the machine translation might become valuable.

We have also evaluated performance of a modification aimed at correction German SMT output. We were
satisfied with the results during model development, achieving results similar to Czech pipeline,
which confirms that the classification tasks themselves (as they were defined in the thesis) are
not language dependent. No special attention to manually modifying the feature set or choosing
different approach was required.

The results of the final evaluation for German were however poor. Aside from the morphology module applied separately,
German MLFix always worsened the MT output as the automatic metric have shown. This was further confirmed
by a manual evaluation. We pointed out
several indications that this might be caused by a poor performance of the German inflection module
we use to generate new wordforms. We still have to investigate the matter further to confirm this hypothesis.
If confirmed, replacing (or improving) the module in the future might be the first and fastest
way to improvement.
Another goal is to investigate the German MT errors more thoroughly and adapt the German pipeline to the findings.

Even though we mainly focused on morphology correction in this thesis, MLFix can be further improved
in the future by introducing other statistical modules addressing additional MT errors. As an example
we mention a possible word reordering model because there were instances where words with new surface
forms predicted by MLFix still needed rearrangement to fully utilize the modification. Due to the modularity
of Treex framework, introducing new improvements to MLFix is easy.


%%% Bibliography
%\include{bibliography}
\bibliographystyle{plainnat}
%\bibliographystyle{obo-bst}
\phantomsection
\addcontentsline{toc}{chapter}{Literature}
\bibliography{bib-varis}

%%% Figures used in the thesis (consider if this is needed)
\listoffigures

%%% Tables used in the thesis (consider if this is needed)
%%% In mathematical theses, it could be better to move the list of tables to the beginning of the thesis.
\listoftables

\appendix

%%% Abbreviations used in the thesis, if any, including their explanation
%%% In mathematical theses, it could be better to move the list of abbreviations to the beginning of the thesis.
%\chapwithtoc{List of Abbreviations}
%\chapwithtoc{List of Abbreviations}
\textbf{SMT} statistical machine translation


%%% Attachments to the master thesis, if any. Each attachment must be
%%% referred to at least once from the text of the thesis. Attachments
%%% are numbered.
%%%
%%% The printed version should preferably contain attachments, which can be
%%% read (additional tables and charts, supplementary text, examples of
%%% program output, etc.). The electronic version is more suited for attachments
%%% which will likely be used in an electronic form rather than read (program
%%% source code, data files, interactive charts, etc.). Electronic attachments
%%% should be uploaded to SIS and optionally also included in the thesis on a~CD/DVD.
%\chapwithtoc{Attachments}
\chapter{Contents of the CD}
\label{attach:cd}

The attached CD contains following items:

\begin{itemize}
\item
    \emph{config} - configuration files for the statistical components
\item
    \emph{data} - sample input data
\item
    \emph{INSTALL} - a file containing a manual for MLFix installation
\item
    \emph{Makefile} - a Makefile containing MLFix commands
\item
    \emph{models} - models used by MLFix components that are currently outside of Treex-shared directory
\item
    \emph{README} - manual for using MLFix
\item
	\emph{scenarios} - scenarios that are not implemented by Treex::Scen:: blocks
\item
	\emph{scripts} - scripts used for minor tasks (e.g. model training, data pre-, post-processing)
\item
	\emph{settings\_cs.mak}, \emph{settings\_de.mak} - settings files for EN-CS and EN-DE MLFix pipeline
\item
	\emph{thesis.pdf} - a PDF file containing this thesis
\end{itemize}

\chapter{MLFix scenarios}
\label{chap:scenario}

In this attachement we list both Czech and German scenarios we used for Depfix processing
pipeline. The scenarios are generated by the %\code{Treex::Scen::MLFix::} blocks, at the beginning
of the scenario there is a commented Treex command which generated the scenario. The scenarios
are listed in the order they are applied to the input. Note that the Czech and German blocks
are called only in their respective scenario.

\section{English-Czech}

\subsection{Analysis on m-layer}

\lstset{basicstyle=\ttfamily,breaklines=true}
\begin{lstlisting}

# Source (English)
# treex -d Scen::MLFix::Analysis_1 language=en iset_driver="en::penn"
Util::SetGlobal language=en selector=
Util::Eval zone='$zone->remove_tree("a") if $zone->has_tree("a");'
W2A::EN::Tokenize
W2A::EN::NormalizeForms
W2A::EN::FixTokenization
W2A::EN::TagMorphoDiTa lemmatize=0
W2A::EN::FixTags
W2A::EN::Lemmatize
A2A::ConvertTags input_driver=en::penn

# treex -d Scen::MLFix::NER language=en model=ner-eng-ie.crf-3-all2008.ser.gz
Util::SetGlobal language=en
A2N::EN::StanfordNamedEntities model=ner-eng-ie.crf-3-all2008.ser.gz
A2N::EN::DistinguishPersonalNames


# Target (Czech)
# treex -d Scen::MLFix::Analysis_1 language=cs tagger=morphodita iset_driver="cs::pdt"
Util::SetGlobal language=cs selector=
Util::Eval zone='$zone->remove_tree("a") if $zone->has_tree("a");'
W2A::CS::Tokenize
W2A::CS::TagMorphoDiTa lemmatize=1
W2A::CS::FixMorphoErrors
W2A::CS::FixGuessedLemmas
A2A::ConvertTags input_driver=cs::pdt
A2N::CS::SimpleRuleNER

# Reference (Czech)
# treex -d Scen::MLFix::Analysis_1 language=cs selector=ref tagger=morphodita iset_driver="cs::pdt"
Util::SetGlobal language=cs selector=ref
Util::Eval zone='$zone->remove_tree("a") if $zone->has_tree("a");'
W2A::CS::Tokenize
W2A::CS::TagMorphoDiTa lemmatize=1
W2A::CS::FixMorphoErrors
W2A::CS::FixGuessedLemmas
A2A::ConvertTags input_driver=cs::pdt
A2N::CS::SimpleRuleNER

\end{lstlisting}

\subsection{Alignment}

\begin{lstlisting}

# English-Czech
# treex -d Scen::MLFix::RunMGiza from_language=cs to_language=en model=cs-en
Align::A::AlignMGiza dir_or_sym=intersection selector= from_language=cs to_language=en model_from_share=cs-en tmp_dir=/mnt/h/tmp cpu_cores=1
Align::AddMissingLinks layer=a selector= language=cs target_language=en alignment_type=intersection
Align::ReverseAlignment language=cs selector=

# Reference
Align::A::MonolingualGreedy selector=T language=cs to_selector=ref

\end{lstlisting}

\subsection{Analysis on a-layer}

\begin{lstlisting}

# Source (English)
# treex -d Scen::MLFix::Analysis_2 language=en parser=mst
Util::SetGlobal language=en selector=
W2A::EN::ParseMST
W2A::EN::SetIsMemberFromDeprel
W2A::EN::RehangConllToPdtStyle
W2A::EN::FixNominalGroups
W2A::EN::FixIsMember
W2A::EN::FixAtree
W2A::EN::FixMultiwordPrepAndConj
W2A::EN::FixDicendiVerbs
W2A::EN::SetAfunAuxCPCoord
W2A::EN::SetAfun


# Target (Czech)
# treex -d Scen::MLFix::Analysis_2 language=cs src_language=en parser=
Util::SetGlobal language=cs selector=
A2A::ProjectTreeThroughAlignment language=en to_language=cs to_selector=

# Reference (Czech)
# treex -d Scen::MLFix::Analysis_2 language=cs selector=ref
Util::SetGlobal language=cs selector=ref
W2A::CS::ParseMSTAdapted
W2A::CS::FixAtreeAfterMcD
W2A::CS::FixIsMember
W2A::CS::FixPrepositionalCase
W2A::CS::FixReflexiveTantum
W2A::CS::FixReflexivePronouns

\end{lstlisting}

\subsection{Fixing}

\begin{lstlisting}

# Preparation
# treex -d Scen::MLFix::FixPrepare src_language=en tgt_language=cs
Util::Eval language=cs selector=T zone='$zone->remove_tree("a") if $zone->has_tree("a");'
Util::Eval language=cs selector=FIXLOG zone='$zone->set_sentence("");'
A2A::CopyAtree source_language=cs language=cs selector=T align=1
Align::AlignForward language=cs selector=T overwrite=0 preserve_type=0

# Fixing
# treex -d Scen::MLFix::Fix "mark_method=scikit-learn "fix_method=scikit-learn" "language=cs" "selector=" "mark_config_file=XXX" "fix_config_file=XXX" "iset_driver=cs::pdt"
MLFix::MarkByScikitLearn language=cs selector= config_file=XXX
MLFix::CS::ScikitLearn language=cs selector= config_file=XXX iset_driver=cs::pdt

\end{lstlisting}

\subsection{Detokenization}

\begin{lstlisting}

# treex -d Scen::MLFix::WriteSentences language=cs
Util::SetGlobal language=cs selector=
A2W::Detokenize
A2W::CS::DetokenizeUsingRules
A2W::CS::DetokenizeDashes
Util::Eval zone='print $zone->sentence . "\n";'

\end{lstlisting}

\section{English-German}

\subsection{Analysis on m-layer}

\begin{lstlisting}

# Source (English)
# treex -d Scen::MLFix::Analysis_1 language=en iset_driver="en::penn"
Util::SetGlobal language=en selector=
Util::Eval zone='$zone->remove_tree("a") if $zone->has_tree("a");'
W2A::EN::Tokenize
W2A::EN::NormalizeForms
W2A::EN::FixTokenization
W2A::EN::TagMorphoDiTa lemmatize=0
W2A::EN::FixTags
W2A::EN::Lemmatize
A2A::ConvertTags input_driver=en::penn

# treex -d Scen::MLFix::NER language=en model=ner-eng-ie.crf-3-all2008.ser.gz
Util::SetGlobal language=en
A2N::EN::StanfordNamedEntities model=ner-eng-ie.crf-3-all2008.ser.gz
A2N::EN::DistinguishPersonalNames


# Target (Czech)
# treex -d Scen::MLFix::Analysis_1 language=de tagger=mate iset_driver="de::conll2009"
Util::SetGlobal language=de selector=
Util::Eval zone='$zone->remove_tree("a") if $zone->has_tree("a");'
W2A::DE::Tokenize
W2A::DE::LemmatizeMate
W2A::DE::ParseMate lemmatize=0
A2A::DE::CoNLL2Iset

# Reference (Czech)
# treex -d Scen::MLFix::Analysis_1 language=de selector=ref tagger=mate iset_driver="de::conll2009"
Util::SetGlobal language=de selector=ref
Util::Eval zone='$zone->remove_tree("a") if $zone->has_tree("a");'
W2A::DE::Tokenize
W2A::DE::LemmatizeMate
W2A::DE::ParseMate lemmatize=0
A2A::DE::CoNLL2Iset

\end{lstlisting}

\subsection{Alignment}

\begin{lstlisting}

# English-German
# treex -d Scen::MLFix::RunMGiza from_language=de to_language=en model=de-en
Align::A::AlignMGiza dir_or_sym=intersection selector= from_language=de to_language=en model_from_share=de-en tmp_dir=/mnt/h/tmp cpu_cores=1
Align::AddMissingLinks layer=a selector= language=de target_language=en alignment_type=intersection
Align::ReverseAlignment language=de selector=

# Reference
# Align::A::MonolingualGreedy selector= language=de to_selector=ref

\end{lstlisting}

\subsection{Analysis on a-layer}

\begin{lstlisting}

# Source (English)
# treex -d Scen::MLFix::Analysis_2 language=en parser=mst
Util::SetGlobal language=en selector=
W2A::EN::ParseMST
W2A::EN::SetIsMemberFromDeprel
W2A::EN::RehangConllToPdtStyle
W2A::EN::FixNominalGroups
W2A::EN::FixIsMember
W2A::EN::FixAtree
W2A::EN::FixMultiwordPrepAndConj
W2A::EN::FixDicendiVerbs
W2A::EN::SetAfunAuxCPCoord
W2A::EN::SetAfun


# Target (German)
# treex -d Scen::MLFix::Analysis_2 language=de src_language=en parser=
Util::SetGlobal language=de selector=
A2A::ProjectTreeThroughAlignment language=en to_language=de to_selector=

# Reference (German)
# treex -d Scen::MLFix::Analysis_2 language=de selector=ref
Util::SetGlobal language=de selector=ref
W2A::DE::ParseMate
A2A::DE::CoNLL2Iset

\end{lstlisting}

\subsection{Fixing}

\begin{lstlisting}

# Preparation
# treex -d Scen::MLFix::FixPrepare src_language=en tgt_language=de
Util::Eval language=de selector=T zone='$zone->remove_tree("a") if $zone->has_tree("a");'
Util::Eval language=de selector=FIXLOG zone=$zone->set_sentence("");
A2A::CopyAtree source_language=de language=de selector=T align=1
Align::AlignForward language=de selector=T overwrite=0 preserve_type=0

# Fixing
# treex -d Scen::MLFix::Fix mark_method=scikit-learn fix_method=scikit-learn language=de selector= mark_config_file=XXX fix_config_file=XXX iset_driver=de::conll2009
MLFix::MarkByScikitLearn language=de selector= config_file=XXX
MLFix::DE::ScikitLearn language=de selector= config_file=XXX iset_driver=de::conll2009

\end{lstlisting}

\subsection{Detokenization}

\begin{lstlisting}

# treex -d Scen::MLFix::WriteSentences language=de
Util::SetGlobal language=de selector=
A2W::Detokenize
Util::Eval zone='print $zone->sentence . "\n";'

\end{lstlisting}



\openright
\end{document}
